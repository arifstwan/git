\documentclass[10pt,a4paper]{extarticle}
\usepackage[latin1]{inputenc}
\usepackage{amsmath}
\usepackage{microtype}
\usepackage[none]{hyphenat}
\usepackage{verbatim}
\usepackage{amsfonts}
\usepackage{amssymb}
\usepackage{enumitem}
\renewcommand{\familydefault}{\sfdefault}
\usepackage{mathpazo}
\renewcommand{\rmdefault}{put}
\usepackage{enumitem}
\usepackage[dvipsnames,svgnames]{xcolor}
\usepackage{tkz-euclide}
\usetkzobj{all}
\usepackage{graphicx}
\usepackage{tikz} 	
\usepackage{adjustbox}
\usepackage{multicol}
\usepackage{lipsum}
\usepackage[left=0.7cm,right=1cm,top=1cm,bottom=1.5cm]{geometry}
\usepackage{cancel} \usepackage{xcolor}
\usepackage{tcolorbox}
\usetikzlibrary{decorations.pathmorphing,patterns}
\usetikzlibrary{decorations.pathreplacing,calc}
 \newcommand\coret[2][red]{\renewcommand\CancelColor{\color{#1}}\cancel{#2}}
\SetLabelAlign{Center}{\hfil\makebox[1.0em]{#1}\hfil}

%%_------= solusi


% Set this =0 to hide, =1 to show

% Set this =0 to hide, =1 to show
\newtcolorbox{mybox}[1][] { colframe = blue!10, colback = blue!3,boxsep=0pt,left=0.2em, coltitle = blue!20!black, title = \textbf{jawab}, #1, } 

%---------- kunci (jika 1 ) muncul
\def\tampilkunci{1}
\newcommand{\hide}[1]{\ifnum\tampilkunci=1
%
\begin{mybox}
 #1
\end{mybox}
%
\vspace{\baselineskip}\fi}



\newcommand*\cicled[1]{\tikz[baseline=(char.base)]{\node[white, shape=circle, fill=red!80,draw,inner sep=0.5pt](char){#1};}}

\newcommand*\kunci[1]{\ifnum\tampilkunci=1
%
\tikz[baseline=(char.base)]{\node[red, shape=circle,draw,inner sep=0.5pt,xshift=2pt](char){#1};}\stepcounter{enumii}
\fi\ifnum\tampilkunci=0
%
\hspace{3pt}#1\stepcounter{enumii}
%
\fi}

\newcommand*\silang[1]{\tikz[baseline=(char.base)]{
\draw[red,thick](-0.2,-0.20)--(0.2,0.2);
\draw[red,thick](-0.2,0.20)--(0.2,-0.2);
\node[black](char){#1};
}}

\newcommand*\centang[1]{\tikz[baseline=(char.base)]{
\draw[red, very thick](-0.2,0.1)--(-0.1,0)--(0.2,0.3);
\node(char){#1};
}}

\newcommand*\merah[1]{
\textcolor{red}{#1}}
\newcommand*\pilgan[1]{
\begin{enumerate}[label=\Alph*., itemsep=0pt,topsep=0pt,leftmargin=*,align=Center] #1 
\end{enumerate}}
\newcommand*\pernyataan[1]{
\begin{enumerate}[label=(\arabic*), itemsep=0pt,topsep=0pt,leftmargin=*] #1 
\end{enumerate}}

\newcommand{\pilgani}[1]{                            \vspace{-0.3cm}\begin{multicols}{2}
 \begin{enumerate}[label=\Alph*., itemsep=0pt,topsep=0pt,leftmargin=*,align=Center]#1                     \end{enumerate}
 \phantom{ini cuma sapi, wedus, dan ayam}
 \end{multicols}}


\begin{document}


 \textbf{Latihan Ulangan Gerak Harmonik} \phantom{ini nama siswa yang aaamengerjakan soal kuis ini }  

\begin{multicols*}{2}\raggedcolumns

\begin{enumerate}
\item Sebuah bola bermassa 100 gram mula-mula diam, dipukul dengan tongkat sehingga kecepatannya menjadi 40 m/s. Besarnya impuls dari gaya pemukul tersebut adalah . . . . 
\pilgani{
   \item 4 Ns
   \item 10 Ns
   \item 40 Ns
   \item 100 Ns
   \item 400 Ns
}
\vspace{1.5cm}

\item Benda bermassa 0,2 kg bergerak dari kiri ke kanan dengan kecepatan 6 m/s kemudian mengenai dinding selama 0,01 sekon dan kecepatannya berubah menjadi 10 m/s. Besar gaya yag diberikan dindin adalah . . . 
\pilgani{
   \item 80 N
   \item 320 N
   \item 8 N
   \item 32 N
   \item 0 N
}
\vspace{1.5cm}

\item  Sebuah benda diberi gaya sebesar 1.000 N dalam waktu singkat, yakni 0,05 s. Jika perubahan yang terjadi adalah dari 2 m/s menjadi 6 m/s. Maka massa dari benda tersebut adalah . . .
\pilgani{
   \item 2,5 kg
   \item 5 kg
   \item 25 kg
   \item 12,5 kg
   \item 6,25 kg
}
\vspace{1.5cm}

\item Bola mendatar dengan massa 10 gram dilemparkan mendatar dengan kecepatan 15 m/s dan memantul dengan kecepatan 5 m/s. Maka impuls yang dikerjakan dinding adalah . . .
\pilgani{
   \item 0,2 Ns
   \item 0,1 Ns
   \item 0,02 Ns
   \item 0,01 Ns
   \item 2 Ns
}
\vspace{1.5cm}

\item Perhatikan gambar berikut! \\
\begin{tikzpicture}
\draw[<-, thick] (0,3) -- (0,0) -- (7,0);
\draw (1,0) -- (2,2) -- (4,2) -- (5,0);
\foreach \x in {1,2,4,5}
{
   \node  at (\x,-0.3){\x};
}
\foreach \y in {1,2}
{
   \node  at (-0.3,\y){\y};
}
\draw[dashed](0,2) --(2,2);
\node at (0, 3.5) {$F$};
\node at (7.5,0){$t$ (s)};
\end{tikzpicture}
Maka besar Impuls pada 3kg benda yang diberi gaya seperti pada grafik tersebut adalah . . .
\pilgani{
   \item 3 Ns
   \item 5 Ns
   \item 6 Ns
   \item 10 Ns
   \item 12 Ns
}
\vspace{2cm}
\item Suatu benda bermassa 10 kg mula-mula bergerak ke arah kanan dengan kecepatan 4 m/s. Setelah dipukul dengan gaya 100 N ke arah sebaliknya, benda tersebut berbalik arah dengan kecepatan 6 m/s. Waktu kontak gaya kepada benda adalah . . .
\pilgani{
   \item 0.1 s
   \item 0.2 s
   \item 0.3 s
   \item 1 s
   \item 2 s
}
\vspace{1.5cm}
\item Benda bermassa 0,2 kg mula-mula diam, kemudian dipukul dengan gaya $F$ selama 0,1 s sehingga kecepatannya 150 m/s. Maka besar gaya $F$ adalah . . 
\pilgani{
   \item 100 N
   \item 200 N
   \item 300 N
   \item 400 N
   \item 500 N
}
\vspace{1.5cm}

\item Benda A bermassa 0,6 kg bergerak mendekatan benda B bermassa 0,4 kg dalam keadaan diam. Benda A melaju dengan kecepatan 2 m/s. Setelah tumbukan, kedua benda menjadi satu. Kelajuan kedua benda adalah . . 
\pilgani{
   \item 1,2 m/s
   \item 2,4 m/s
   \item 3,6 m/s
   \item 4,8 m/s
   \item 5,4 m/s
   }
   \vspace{1.5cm}

\item Dua buah bola A dan B bermassa sama. Bola A bergerak ke kanan dengan kecepatan 30 m/s dan bola B 10 m/s ke kiri. Terjadi tumbukan lenting sempurna, maka kecepatan bola A dan B setelah tumbukan adalah . . .
\pilgan{
   \item 30 m/s kiri dan 10 m/s kanan
   \item 10 m/s kiri dan 30 m/s kanan
   \item 10 m/s kanan dan 30 m/s kiri
   \item 20 m/s kanan dan 20 m/s kiri
   \item 20 m/s kiri dan 20 m/s kanan
}
\vspace{1cm}
\item 

\end{enumerate}
\end{multicols*}\end{document}







\documentclass[10pt,a4paper]{article}
\usepackage[latin1]{inputenc}
\usepackage{amsmath}
\usepackage{microtype}
\usepackage[none]{hyphenat}
\usepackage{verbatim}
\usepackage{amsfonts}
\usepackage{amssymb}
\usepackage{enumitem}
\renewcommand{\familydefault}{\sfdefault}
\usepackage{mathpazo}
\renewcommand{\rmdefault}{put}
\usepackage{enumitem}
\usepackage[dvipsnames,svgnames]{xcolor}
\usepackage{tkz-euclide}
\usetkzobj{all}
\usepackage{graphicx}
\usepackage{tikz} 	
\usepackage{adjustbox}
\usepackage{multicol}
\usepackage{lipsum}
\usepackage[left=0.1cm,right=0.7cm,top=0.2cm,bottom=1.5cm]{geometry}
\usepackage{cancel} \usepackage{xcolor}
\usepackage{tcolorbox}
\usetikzlibrary{decorations.pathmorphing,patterns}
\usetikzlibrary{decorations.pathreplacing,calc}
 \newcommand\coret[2][red]{\renewcommand\CancelColor{\color{#1}}\cancel{#2}}
\SetLabelAlign{Center}{\hfil\makebox[1.0em]{#1}\hfil}

%%_------= solusi


% Set this =0 to hide, =1 to show

% Set this =0 to hide, =1 to show
\newtcolorbox{mybox}[1][] { colframe = blue!10, colback = blue!3,boxsep=0pt,left=0.2em, coltitle = blue!20!black, title = \textbf{jawab}, #1, } 

%---------- kunci (jika 1 ) muncul
\def\tampilkunci{1}
\newcommand{\hide}[1]{\ifnum\tampilkunci=1
%
\begin{mybox}
 #1
\end{mybox}
%
\vspace{\baselineskip}\fi}



\newcommand*\cicled[1]{\tikz[baseline=(char.base)]{\node[white, shape=circle, fill=red!80,draw,inner sep=0.5pt](char){#1};}}

\newcommand*\kunci[1]{\ifnum\tampilkunci=1
%
\tikz[baseline=(char.base)]{\node[red, shape=circle,draw,inner sep=0.5pt,xshift=2pt](char){#1};}\stepcounter{enumii}
\fi\ifnum\tampilkunci=0
%
\hspace{3pt}#1\stepcounter{enumii}
%
\fi}

\newcommand*\silang[1]{\tikz[baseline=(char.base)]{
\draw[red,thick](-0.2,-0.20)--(0.2,0.2);
\draw[red,thick](-0.2,0.20)--(0.2,-0.2);
\node[black](char){#1};
}}

\newcommand*\centang[1]{\tikz[baseline=(char.base)]{
\draw[red, very thick](-0.2,0.1)--(-0.1,0)--(0.2,0.3);
\node(char){#1};
}}

\newcommand*\merah[1]{
\textcolor{red}{#1}}
\newcommand*\pilgan[1]{
\begin{enumerate}[label=\Alph*., itemsep=0pt,topsep=0pt,leftmargin=*,align=Center] #1 
\end{enumerate}}
\newcommand*\pernyataan[1]{
\begin{enumerate}[label=(\arabic*), itemsep=0pt,topsep=0pt,leftmargin=*] #1 
\end{enumerate}}

\begin{document}

\setlength{\abovedisplayskip}{0pt}
\setlength{\belowdisplayskip}{3pt}
\setlength{\abovedisplayshortskip}{0pt}
\setlength{\belowdisplayshortskip}{3pt}
%-----------------------------------------------

 \centering
  \renewcommand{\arraystretch}{2}
  \begin{tabular}{  |>{\centering\arraybackslash}m{4cm}|%
                    >{\centering\arraybackslash}m{11cm}|%
                    >{\centering\arraybackslash}m{4cm}|%
  }
    \hline
    \vspace{0.15cm} 
    \tikz[baseline=(char.base)]{
\draw[green!80!black](-0.3,-0.2) rectangle (0.3,0.2);
\node[green](char){line};
} \small{ arifstwan} &       \textbf{Soal Modul Termodinamika } 
          &  arif.stwan@gmail.com
  \\ \hline 
    
  \end{tabular}
\setlength{\columnsep}{0pt}
\vspace{0.15cm}

\begin{multicols*} {3} 
\newcommand{\tikzmark}[2]{\tikz[remember picture,baseline=(#1.base)]{\node[inner sep=0pt] (#1) {#2};}} 


\begin{enumerate}[itemsep=0mm]

%------------ nomor 1----------
\item Tentukan bagian yang dicari dari perbandingan berikut!
\begin{enumerate}[label=\alph*)]
\item $ \left(\frac{3}{T_1}\right)^2=\left(\frac{1}{9}\right)^3$
\item $\left(\frac{T_2}{8}\right)^2=\left(\frac{3}{12}\right)^3$
\item
$\left(\frac{1}{27} \right)^2=\left(\frac{R_2}{18}\right)^3$
\end{enumerate}

\item Gunakan cara $$ x=\frac{d\sqrt{m_1}}{\sqrt{m_1}+\sqrt{m_2}}$$ 
dimana

$d$: jarak dua benda awal 

$x$ : jarak dari yang ditanyakan dari $m_1$

$m_1, m_2$ : massa dua benda awal.
\begin{enumerate}[label=\alph*)]

\item Dua buah benda bermassa 4kg dan 9 kg berada pada jarak 15 m. Ada benda bermassa 2,78 kg diletakkan di antara kedua benda. Berapakah jarak benda ketiga (terakhir) dari benda 4 kg, jika besar gaya yang dirasakan adalah NOL . . . .
\item Dua buah benda A dan B bermassa 81 juta kg dan 144 juta kg dipisahkan pada jarak 42 juta km. Di manakah letak agar \textbf{medan gravitasi} totalnya NOL ? . . . . .
\end{enumerate}
\end{enumerate}   

Pembahasan 
\begin{enumerate}[itemsep=0mm]
\item 
\begin{enumerate}[label=\alph*)]
\item Karena yang dihitung adalah ruas kiri, pastikan tidak ada pangkatnya yang kiri
\begin{align*} \left(\frac{3}{T_1}\right)^2&=\left(\frac{1}{9}\right)^3 \\
\sqrt{\left(\frac{3}{T_1}\right)^2}&=\sqrt{\left(\frac{1}{9}\right)^3} \\
\left(\frac{3}{T_1}\right)&=\left(\frac{1}{3}\right)^3 \\
\left(\frac{3}{T_1}\right)&=\left(\frac{1}{27}\right) \\
\left(\frac{3}{1}\right)&=\left(\frac{T_1}{3}\right) \\
T_1&=3
\end{align*}
\item \begin{align*}
\left(\frac{T_2}{8}\right)^2&=\left(\frac{3}{12}\right)^3\\
\sqrt{\left(\frac{T_2}{8}\right)}^2&=\sqrt{\left(\frac{3}{12}\right)^3}\\
\left(\frac{T_2}{8}\right)&=\sqrt{\left(\frac{\coret{3}^1}{\coret{12}^4}\right)^3}\\
\left(\frac{T_2}{8}\right)&=\sqrt{\left(\frac{1}{4}\right)^3}\\
\left(\frac{T_2}{8}\right)&=\left(\frac{1}{2}\right)^3\\
\left(\frac{T_2}{8}\right)&=\left(\frac{1}{8}\right)\\
T_2 &=1\\
\end{align*}

\item
\begin{align*}



\left(\frac{1}{27} \right)^2=\left(\frac{R_2}{18}\right)^3$
\end{enumerate}
\end{enumerate}
\end{multicols*}

 \end{document}

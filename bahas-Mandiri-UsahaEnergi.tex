\documentclass[10pt,a4paper]{article}
\usepackage[latin1]{inputenc}
\usepackage{amsmath}
\usepackage{microtype}
\usepackage[none]{hyphenat}
\usepackage{verbatim}
\usepackage{amsfonts}
\usepackage{amssymb}
\usepackage{enumitem}
\renewcommand{\familydefault}{\sfdefault}
\usepackage{mathpazo}
\renewcommand{\rmdefault}{put}
\usepackage{enumitem}
\usepackage[dvipsnames,svgnames]{xcolor}
\usepackage{tkz-euclide}
\usetkzobj{all}
\usepackage{graphicx}
\usepackage{fancyhdr}
\usepackage{tikz} 	
\usepackage{adjustbox}
\usepackage{pgfplots}
\usepackage{multicol}
\usepackage{lipsum}
\usepackage[left=0.7cm,right=1cm,top=1cm,bottom=1.5cm]{geometry}
\usepackage{cancel} \usepackage{xcolor}
\usepackage{tcolorbox}
\usetikzlibrary{decorations.pathmorphing,patterns}
\usetikzlibrary{decorations.pathreplacing,calc}
 \newcommand\coret[2][red]{\renewcommand\CancelColor{\color{#1}}\cancel{#2}}
\SetLabelAlign{Center}{\hfil\makebox[1.0em]{#1}\hfil}

%%_------= solusi


% Set this =0 to hide, =1 to show

% Set this =0 to hide, =1 to show
\newtcolorbox{mybox}[1][] { colframe = blue!10, colback = blue!3,boxsep=0pt,left=0.2em, coltitle = blue!20!black, title = \textbf{jawab}, #1, } 

%---------- kunci (jika 1 ) muncul
\def\tampilkunci{1}
\newcommand{\hide}[1]{\ifnum\tampilkunci=1
%
\begin{mybox}
 #1
\end{mybox}
%
\vspace{\baselineskip}\fi\ifnum\tampilkunci=0
%
%\vspace{2cm}
%
\fi}



\newcommand*\cicled[1]{\tikz[baseline=(char.base)]{\node[white, shape=circle, fill=red!80,draw,inner sep=0.5pt](char){#1};}}

\newcommand*\kunci[1]{\ifnum\tampilkunci=1
%
\tikz[baseline=(char.base)]{\node[red, shape=circle,draw,inner sep=0.5pt,xshift=2pt](char){#1};}\stepcounter{enumii}
\fi\ifnum\tampilkunci=0
%
\hspace{3pt}#1\stepcounter{enumii}
%
\fi}

\newcommand*\silang[1]{\tikz[baseline=(char.base)]{
\draw[red,thick](-0.2,-0.20)--(0.2,0.2);
\draw[red,thick](-0.2,0.20)--(0.2,-0.2);
\node[black](char){#1};
}}

\newcommand*\centang[1]{\tikz[baseline=(char.base)]{
\draw[red, very thick](-0.2,0.1)--(-0.1,0)--(0.2,0.3);
\node(char){#1};
}}

\newcommand*\merah[1]{
\textcolor{red}{#1}}
\newcommand*\pilgan[1]{
\begin{enumerate}[label=\Alph*., itemsep=0pt,topsep=0pt,leftmargin=*,align=Center] #1 
\end{enumerate}}
\newcommand*\pernyataan[1]{
\begin{enumerate}[label=(\arabic*), itemsep=0pt,topsep=0pt,leftmargin=*] #1 
\end{enumerate}}
\pagestyle{fancy}
\renewcommand{\headrulewidth}{0pt}
\rfoot{\tiny{arifstwan}}
\newcommand{\pilgani}[1]{                            \vspace{-0.3cm}\begin{multicols}{2}
 \begin{enumerate}[label=\Alph*., itemsep=0pt,topsep=0pt,leftmargin=*,align=Center]#1                     \end{enumerate}
 \phantom{ini cuma sapi, wedus, dan ayam}
 \end{multicols}}


\begin{document}


 \textbf{Mandiri-Usaha dan Energi} \phantom{ini nama siswa yang aaamengerjakan soal kuis ini }  

No callculator allowed !  
\begin{multicols*}{2}
\begin{enumerate}
\setcounter{enumi}{25}
% no26 ---------------------------------------------------------
\item Benda A dan B bermassa sama jatuh dari ketinggian yang berbeda yaitu $h$ dan $2h$. Jika A menyentuh permukaan tanah dengan kecepatan v, benda B akan menyentuh permukaan tanah dengan energi kinetik sebesar . . . . 
\pilgani{
	\item $\frac{1}{4}mv^2$
	\item $\frac{1}{2}mv^2$
	\item $\frac{3}{4}mv^2$
	\item[\kunci{D.}] $mv^2$
	\item $\frac{3}{2}mv^2$
	}
\hide{
	Hitung $EK_A$ dengan 
	\begin{align*}
	EM_1&=EM_1'\\
	mgh &= EK_A+EP_A\\
	mgh &=\frac{1}{2}mv^2 
	\end{align*}
	Sedangkan $EK_B$
	\begin{align*}
	EM_1&=EM_1'\\
	mg(2h) &= EK_B+EP_B\\
	2mgh &=EK_B + 0\\
	2.\left(\frac{1}{2}mv^2\right) &=EK_B\\
	mv^2 &= EK_B
	\end{align*}
	Jadi karena tingginya 2 kali maka energi kinetiknya 2 kali enegi kinetik A. Yakni 2 kali $\frac{1}{2}mv^2$
	}

% no27 -------------------------------------------
\item Dua benda masing-masing massa $m_1$ dan $m_2$ yang berbeda. Jika kedua benda mempunyai energi kinetik yang sama, kedua benda juga mempunyai . . . .
\pilgani{
	\item kecepatan yang sama
	\item momentum yang sama
	\item percepatan yang sama
	\item [\kunci{D.}] momentum yang sama
	\item gaya yang sama
	} 
\hide{
 Kita bandingkan, apakah energi kecepatannya sama?
 \begin{align*}
 EK_1 &= EK_2\\
 \frac{1}{2} m_1v_1^2 &= \frac{1}{2} m_2v_2^2
 \end{align*}
 Apakah $v_1$ dan $v_2$ sama? JELAS BERBEDA \\
 Apakah percepatan sama?\\
  Percepatan tidak dapat diketahui tanpa diketahui gaya yang bekerja
 Momentum $p = mv$. Apakah $m_1v_1$ sama $m_2v_2$ ? jika $m_1.v_1^2=m_2.v_2^2$ maka  $m_1v_1 \neq m_2v_2$ \\
 Jadi momentumnyaTIDAK SAMA	
}

% no28 -----------------------------------------
\item Sebuah mobil bermassa $m$ memiliki mesin berdaya $P$. Jika pengaruh gaya gesek kecil, waktu minimum yang diperlukan agar mencapai kecepatan $v$ dari keadaan diamnya adalah . . .
\pilgani{
	\item $\frac{mv}{P}$
	\item $\frac{P}{mv}$
	\item $\frac{2P}{mv^2}$
	\item $\frac{P}{mv^2}$
	\item[\kunci{E.}] $\frac{mv^2}{2P}$
	}
\hide{ Persamaan daya ada hubungannya dengan waktu
	\begin{align*}
	P&=\frac{W}{t}=\frac{\Delta E}{t}=\frac{\frac{1}{2}mv^2-0}{t}\\
	t&=\frac{mv^2}{2P}
	\end{align*}
	}
	
% no29 -------------------------------------
\item Sebuah balok ditarik dengan gaya 100 N yang membentuk sudut 37$^o$ terhadap arah mendatar. Besar usaha yang dilakukan oleh gaya untuk berpindah sejauh 5 m adalah . . . 
\pilgani{
	\item 100 J
	\item 200 J
	\item 300 J
	\item[\kunci{D.}] 400 J
	\item 500 J
	}
\hide{
        Menghitung Usaha gunakan
        \begin{align*}
        W&=F \cos \theta s = 100 \cos 37^o 5 = 400 J
        \end{align*}}




% no30 -----------------------------------
\item Sebuah benda bermassa 2  kg mula-mula bergerak dengan kecepatan 72 km/jam. Setelah bergerak sejauh 400 m, kecepatanbenda menjadi 144 km/jam. Usaha total yang dilakukan benda tersebut jika $g$ = 10 m/s$^2$ adalah . . .
\pilgani{
        \item 20 J
        \item 60 J
        \item 1200 J
        \item 2000 J
        \item 2400 J
        }
\hide{
        Usaha defiisinya adalah perubahan energi kinetik atau energi potensial maka
        \begin{align*}
        W &=\Delta Ek \\
        W &=\frac{1}{2}mv^2 (v_2^2-v_1^2)\\
        W&=\frac{1}{2}2(20^2-10^2)\\
        W&=1200 \text{ J}
        \end{align*}
}

%no31 ----------------------------------
\item Sebuah bola bermassa 1 kg dijatuhkan tanpa kecepatan awal dari atas gedung, bola meluncur melewati jendela A di lantai atas ke jendela B di lantai bawah dengan beda tinggi 2,5 m. Usaha perpindahan bola dari jendela A ke jendela B adalah . . .
\pilgani{
        \item 5 J
        \item 15 J
        \item 20 J
        \item [\kunci{D.}] 25 J
        \item 50 J
        }
\hide{
Usaha adalah perubahan energi. pada kasus ini adalah perubahan energi potensial
\begin{align*}
W &= \Delta EP = mgh_2-mgh_1=mg(\Delta h)=1\times10 \times 2,5 = 25 \text{ J}
\end{align*}
}


\item Untung meregangnkan sebuah pegas sejauh 5 cm butuh gaya 20 N. Energi potensial ketika meregang 10 cm adalah . . . 
\pilgani{
        \item 2 J
        \item 4 J
        \item 10 J
        \item 50 J
        \item 100 J }
\hide{
Dalam satu kalimat digunakan untuk menghitung $k$

$k=\frac{F}{\Delta x}=\frac{20}{5cm}=\frac{20}{0,05}=400 \text{ N/m}$

Energi potensial
\begin{align*}
EP_p &= \frac{1}{2} k\Delta x^2 =\frac{1}{2}400(0,1)^2= 2 \text{ J}
\end{align*}
}

\item Mobil massa 1000 kg bergerak dengan kecepatan 20 m/s dalam arah horizontal. Tiba-tiba pengemudi mengurangi kecepatan mobil menjadi 10 m/s. Usaha yang dilakukan adalah . . .
\pilgani{
        \item [\kunci{A.}]15 $\times 10^4$ J
        \item 30 $\times 10^4$ J
        \item 45 $\times 10^4$ J
        \item 60 $\times 10^4$ J
        \item 75 $\times 10^4$ J
}

\hide{
\begin{align*}
W &=\Delta EK = \frac{1}{2} m(v_2^2-v_1^2)=\frac{1}{2}.1000.(400-100)=15\times 10^4
\end{align*}
}

\item Sebuah peluru bermassa 100 g ditembakkan dengan kecepatan awal 40 m/s dan sudut elevasi 30$^o$, maka besar energi kinetik di titik tertinggi adalah . . .
\pilgani{
        \item nol
        \item 60 J
        \item 120 J
        \item 150 J
        \item 200 J
        }
 \hide{
 Pertanyaan tentang kecepatan, ditanyakan energi kinetik di titik tertinggi (h -> EP), kedua jenis energi disebutkan dalam soal, maka gunakan kekekalan energi mekanik
 \begin{align*}
EM_1 &=  EM_2\\
mgh_1 + \frac{1}{2}mv_1^2 &= mgh_2 + EK_2
 \end{align*}

Karena pada titik tertinggi benda secara vertikal diam, tapi secara horizontal tetap bergerak, yakni tetap seperti kecepatan awal arah x

$v_o = 40 m/s$

$v_{ox}= vo\cos (30) = 40\times \frac{1}{2}\sqrt{3}=20\sqrt{3}$

Jadi energi kinetik di titik tertinggi $EK_2=\frac{1}{2}mv^2=\frac{1}{2}0,1.(20\sqrt{3})^2=60$ J

}



\item 

\hide{
\begin{align*}
W &= \vec{F}\cdot\vec{s}=(2\hat{i}+3\hat{j})\cdot(4\hat{i} + a\hat{j}\\
26 &= 2\times 4 + 3\times a\\
a &= 6
\end{align*}
}

\item 

\hide{gunakan kekekalan energi mekanik saja
\begin{align*}
EM &= EM_2\\
mgh + 0 &= mgh_2 + EK_2\\
mg.20 &= mg.5 + EK_2\\
mg.20&= mg.5 + mg.15
\end{align*}
Jadi perbandingan energi potensial dan kinetik adalah 5 : 15 = 1 : 3
}

\item 

\hide{ daya adalah P
\begin{align*}
P &= \frac{W}{t}=\frac{\Delta EK}{t}\\
P &=\frac{ \frac{1}{2}2000(20^2-10^2)}{10}=30.000
\end{align*}
}

\item 

\hide{
Usaha dalah perubahan energi kinetik
\begin{align*}
W &= \Delta EK = \frac{1}{2}2000(v_2^2-v_1^2)\\
W &= 300000=3\times 10^5 \text{ J}
\end{align*}
}


\item 

\hide{
Bandul naik berapa? naiknya bandul adalah setinggi $\Delta h = l-l\cos 60^o = 0,625$ m

kelajuan saat titik terendah adalah kelajuan di tketinggian 0

\begin{align*}
EM &= EM
mgh+ 0 &= mgh+\frac{1}{2}mv^2\\
10.0,625 &=g.0 + \frac{1}{2}v^2\\
6,25 &= \frac{1}{2}v^2\\
v&=sqrt{12,5} =3,5 \text{ m/s} 
\end{align*}
}


\item 

\hide{
Grafik digunakan untuk mencari $k$
$$k = \frac{F}{\Delta x}=\frac{4}{0,08}=50$$
\begin{align*}
EP_{text{pegas}}&=\frac{1}{2} 50 (0,08)^2=0,16 \text{ J}
\end{align*}
}


\item 

\hide{
Benda beratnya 10 N artinya massa 1 kg. Bergerak mendatar sehingga gaya yang dipakai yang arah horizontal 
\begin{align*}
W&=F \cos 60 s = 40 \frac{1}{2} 10 = 200\text{ J}
\end{align*}
}

\item 

\hide{
Usaha adalah perubahan energi, pada soal ini adalah energi potensial pegas
\begin{align*}
W&=\frac{1}{2}k(\Delta x)^2=\frac{1}{2} \frac{F}{\coret{\Delta x}}(\Delta x)^{\coret{2}}\\
W &= 2,25 \text{ J}
\end{align*}
}


\item 

\hide{
usaha secara vertikal artinya perubahan energi potensial
\begin{align*}
W &= \Delta mgh = mg(\Delta h)\\
150 &=2.10(\Delta h)\\
7,5 &= \Delta h
\end{align*}}


\item 

\hide{
Pengertian usaha: usaha adalah hasil kali gaya dan perpindahan yang menghasilkan perubahan energi

Pada soal ini terjadi perubahan energi kinetik, isebabkan oleh gaya gesek
\begin{align*}
W &= \Delta EK\\
F.s &= \frac{1}{2}m(v_2^2-v_1^2)\\
F&= 4800 \text{ N}
\end{align*}
}


\item 

\hide{
\begin{align*}
W &= \Delta EK = \frac{1}{2}2(14^2-10^2)=96\text{ J}
\end{align*}
}


\item 

\hide{
Usaha adalah luas arsiran grafik di atas dikurangi di bawah sumbu x

Karena pada gambar ada di atas semua, maka usahanya adalah 46 J
}

\item 

\hide{
Ada kata ketinggian, ditanyakan kecepatan. Sehingga yang dimaksud adalah kekekalan $EM$

Saat tinggi maksimal tidak ada kecepatan sehingga
$EM = EM$ menjadi $ EK= mgh$
\begin{align*}
EM & = EM\\
\frac{1}{2}mv_1^2 &=EK_2 + \frac{1}{4}mgh\\
mgh &= EK_2 + \frac{1}{4}mgh\\
EK_2&=\frac{3}{4}mgh\\
\frac{1}{2}mv_2^2&=\frac{3}{4}mgh\\
\frac{1}{2}v_2^2&=\frac{3}{4}\left ( \frac{1}{2}mv_1^2 \right)\\
v_2 &= 10\sqrt{3}\text{ m/s}
\end{align*}
}


\item 
\hide{
\begin{align*}
EK_A &: EK_B\\
\frac{1}{2}(0,5)(v)^2 &: \frac{1}{2}(1)(3v)^2\\
0,5 &: 9\\
1 &: 18
\end{align*}
}


\item 
\hide{
\begin{align*}
W &= \Delta EK\\
F.s &= \frac{1}{2}m(0-4^2)\\
F.10 &=\frac{1}{2}50.16\\
F &= 40 \text { N}
\end{align*}
}


\item
\hide{
Pada saat awal tingginya 3R, pada saat akhir tingginya 2R (jari-jari lingkaran). Maka kecepatan bola meninggalkan bisa dicari
\begin{align*}
EM &= EM\\
mg3R &= \frac{1}{2}mv^2 + mg2R\\
mgR &= \frac{1}{2}mv^2\\
v&=\sqrt{2gR}= 3 \text{ m/s}
\end{align*}
}

\item 
\hide{
Ada perubahan ketinggian, dan ditanya kecepatan maka gunakan $EM = EM$
\begin{align*}
EM_1 &= EM_2\\
mgh_1 + 0 &= mgh_2 +\frac{1}{2}mv_2^2\\
mg(\Delta h) &= \frac{1}{2}mv_2^2\\
2g\Delta h &=v^2\\
20(25-5) &v^2\\
20 \text{ m/s}&= v_2
\end{align*}
}


\item 
\hide{
\begin{align*}
P &=\frac{W}{t}\\
500 &=\frac{F.400}{16}\\
F &= 20 \text{ N}
\end{align*} 
}

\item 
\hide{
cari dulu $\Delta x$ untuk menentukan $k$
\begin{align*}
k &=\frac{F}{\Delta x}=\frac{0,6}{0,29-0,27}=30
\end{align*}
Lalu cari energi potensial saat 10 cm
\begin{align*}
EP_p &= \frac{1}{2} 30 (0,1)^2=0,15\text{ J}
\end{align*}
}


\item 
\hide{
Untuk menghitung usaha dengan menghitung luas arsiran (grafik) di atas sumbu x dikurangi di bawah sumbu x , totalnya 110 Joule}

\item
\hide{
faktor yang mempengaruhi besarnya usaha adalah gaya dan sudutnya yang dibentuk}

\end{enumerate}
\end{multicols*} \end{document} 


\documentclass[10pt,a4paper]{article}
\usepackage[latin1]{inputenc}
\usepackage{amsmath}
\usepackage{microtype}
\usepackage[none]{hyphenat}
\usepackage{verbatim}
\usepackage{amsfonts}
\usepackage{amssymb}
\usepackage{enumitem}
\renewcommand{\familydefault}{\sfdefault}
\usepackage{mathpazo}
\renewcommand{\rmdefault}{put}
\usepackage{enumitem}
\usepackage[dvipsnames,svgnames]{xcolor}
\usepackage{tkz-euclide}
\usetkzobj{all}
\usepackage{graphicx}
\usepackage{fancyhdr}
\usepackage{tikz} 	
\usepackage{adjustbox}
\usepackage{pgfplots}
\usepackage{multicol}
\usepackage{lipsum}
\usepackage[left=0.7cm,right=1cm,top=1cm,bottom=1.5cm]{geometry}
\usepackage{cancel} \usepackage{xcolor}
\usepackage{tcolorbox}
\usetikzlibrary{decorations.pathmorphing,patterns}
\usetikzlibrary{decorations.pathreplacing,calc}
 \newcommand\coret[2][red]{\renewcommand\CancelColor{\color{#1}}\cancel{#2}}
\SetLabelAlign{Center}{\hfil\makebox[1.0em]{#1}\hfil}

%%_------= solusi


% Set this =0 to hide, =1 to show

% Set this =0 to hide, =1 to show
\newtcolorbox{mybox}[1][] { colframe = blue!10, colback = blue!3,boxsep=0pt,left=0.2em, coltitle = blue!20!black, title = \textbf{jawab}, #1, } 

%---------- kunci (jika 1 ) muncul
\def\tampilkunci{1}
\newcommand{\hide}[1]{\ifnum\tampilkunci=1
%
\begin{mybox}
 #1
\end{mybox}
%
\vspace{\baselineskip}\fi\ifnum\tampilkunci=0
%
%\vspace{2cm}
%
\fi}



\newcommand*\cicled[1]{\tikz[baseline=(char.base)]{\node[white, shape=circle, fill=red!80,draw,inner sep=0.5pt](char){#1};}}

\newcommand*\kunci[1]{\ifnum\tampilkunci=1
%
\tikz[baseline=(char.base)]{\node[red, shape=circle,draw,inner sep=0.5pt,xshift=2pt](char){#1};}\stepcounter{enumii}
\fi\ifnum\tampilkunci=0
%
\hspace{3pt}#1\stepcounter{enumii}
%
\fi}

\newcommand*\silang[1]{\tikz[baseline=(char.base)]{
\draw[red,thick](-0.2,-0.20)--(0.2,0.2);
\draw[red,thick](-0.2,0.20)--(0.2,-0.2);
\node[black](char){#1};
}}

\newcommand*\centang[1]{\tikz[baseline=(char.base)]{
\draw[red, very thick](-0.2,0.1)--(-0.1,0)--(0.2,0.3);
\node(char){#1};
}}

\newcommand*\merah[1]{
\textcolor{red}{#1}}
\newcommand*\pilgan[1]{
\begin{enumerate}[label=\Alph*., itemsep=0pt,topsep=0pt,leftmargin=*,align=Center] #1 
\end{enumerate}}
\newcommand*\pernyataan[1]{
\begin{enumerate}[label=(\arabic*), itemsep=0pt,topsep=0pt,leftmargin=*] #1 
\end{enumerate}}
\pagestyle{fancy}
\renewcommand{\headrulewidth}{0pt}
\rfoot{\tiny{arifstwan}}
\newcommand{\pilgani}[1]{                            \vspace{-0.3cm}\begin{multicols}{2}
 \begin{enumerate}[label=\Alph*., itemsep=0pt,topsep=0pt,leftmargin=*,align=Center]#1                     \end{enumerate}
 \phantom{ini cuma sapi, wedus, dan ayam}
 \end{multicols}}


\begin{document}


 \textbf{Mandiri-Usaha dan Energi} \phantom{ini nama siswa yang aaamengerjakan soal kuis ini }  

No callculator allowed !  
\begin{multicols*}{2}
\begin{enumerate}
\setcounter{enumi}{25}
% no26 ---------------------------------------------------------
\item Benda A dan B bermassa sama jatuh dari ketinggian yang berbeda yaitu $h$ dan $2h$. Jika A menyentuh permukaan tanah dengan kecepatan v, benda B akan menyentuh permukaan tanah dengan energi kinetik sebesar . . . . 
\pilgani{
	\item $\frac{1}{4}mv^2$
	\item $\frac{1}{2}mv^2$
	\item $\frac{3}{4}mv^2$
	\item[\kunci{D.}] $mv^2$
	\item $\frac{3}{2}mv^2$
	}
\hide{
	Hitung $EK_A$ dengan 
	\begin{align*}
	EM_1&=EM_1'\\
	mgh &= EK_A+EP_A\\
	mgh &=\frac{1}{2}mv^2 
	\end{align*}
	Sedangkan $EK_B$
	\begin{align*}
	EM_1&=EM_1'\\
	mg(2h) &= EK_B+EP_B\\
	2mgh &=EK_B + 0\\
	2.\left(\frac{1}{2}mv^2\right) &=EK_B\\
	mv^2 &= EK_B
	\end{align*}
	Jadi karena tingginya 2 kali maka energi kinetiknya 2 kali enegi kinetik A. Yakni 2 kali $\frac{1}{2}mv^2$
	}

% no27 -------------------------------------------
\item Dua benda masing-masing massa $m_1$ dan $m_2$ yang berbeda. Jika kedua benda mempunyai energi kinetik yang sama, kedua benda juga mempunyai . . . .
\pilgani{
	\item kecepatan yang sama
	\item momentum yang sama
	\item percepatan yang sama
	\item [\kunci{D.}] momentum yang sama
	\item gaya yang sama
	} 
\hide{
 Kita bandingkan, apakah energi kecepatannya sama?
 \begin{align*}
 EK_1 &= EK_2\\
 \frac{1}{2} m_1v_1^2 &= \frac{1}{2} m_2v_2^2
 \end{align*}
 Apakah $v_1$ dan $v_2$ sama? JELAS BERBEDA \\
 Apakah percepatan sama?\\
  Percepatan tidak dapat diketahui tanpa diketahui gaya yang bekerja
 Momentum $p = mv$. Apakah $m_1v_1$ sama $m_2v_2$ ? jika $m_1.v_1^2=m_2.v_2^2$ maka  $m_1v_1 \neq m_2v_2$ \\
 Jadi momentumnyaTIDAK SAMA	
}

% no28 -----------------------------------------
\item Sebuah mobil bermassa $m$ memiliki mesin berdaya $P$. Jika pengaruh gaya gesek kecil, waktu minimum yang diperlukan agar mencapai kecepatan $v$ dari keadaan diamnya adalah . . .
\pilgani{
	\item $\frac{mv}{P}$
	\item $\frac{P}{mv}$
	\item $\frac{2P}{mv^2}$
	\item $\frac{P}{mv^2}$
	\item $\frac{mv^2}{2P}$
	}
\hide{ Persamaan daya ada hubungannya dengan waktu
	\begin{align*}
	P&=\frac{W}{t}=\frac{\Delta E}{t}=\frac{\frac{1}{2}mv^2-0}{t}\\
	t&=\frac{mv^2}{2P}
	\end{align*}
	}
	
% no29 -------------------------------------
\item Sebuah balok ditarik dengan gaya 100 N yang membentuk sudut 37$^o$ terhadap arah mendatar. Besar usaha yang dilakukan oleh gaya untuk berpindah sejauh 5 m adalah . . . 
\pilgani{
	\item 100 J
	\item 200 J
	\item 300 J
	\item 400 J
	\item 500 J
	}
\hide{
        Menghitung Usaha gunakan
        \begin{align*}
        W&=F \cos \theta s = 100 \cos 37^o 5 = 400 J
        \end{align*}}




% no30 -----------------------------------
\item 
\end{enumerate}
\end{multicols*} \end{document} 


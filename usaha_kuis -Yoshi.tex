\documentclass[11pt,a4paper]{article}
\usepackage[latin1]{inputenc}
\usepackage{amsmath}
\usepackage{microtype}
\usepackage[none]{hyphenat}
\usepackage{verbatim}
\usepackage{amsfonts}
\usepackage{amssymb}
\usepackage{enumitem}
\renewcommand{\familydefault}{\sfdefault}
\usepackage{mathpazo}
\renewcommand{\rmdefault}{put}
\usepackage{enumitem}
\usepackage[dvipsnames,svgnames]{xcolor}
\usepackage{tkz-euclide}
\usetkzobj{all}
\usepackage{graphicx}
\usepackage{tikz} 	
\usepackage{adjustbox}
\usepackage{multicol}
\usepackage{lipsum}
\usepackage[left=0.7cm,right=1cm,top=1cm,bottom=1.5cm]{geometry}
\usepackage{cancel} \usepackage{xcolor}
\usepackage{tcolorbox}
\usetikzlibrary{decorations.pathmorphing,patterns}
\usetikzlibrary{decorations.pathreplacing,calc}
 \newcommand\coret[2][red]{\renewcommand\CancelColor{\color{#1}}\cancel{#2}}
\SetLabelAlign{Center}{\hfil\makebox[1.0em]{#1}\hfil}

%%_------= solusi


% Set this =0 to hide, =1 to show

% Set this =0 to hide, =1 to show
\newtcolorbox{mybox}[1][] { colframe = blue!10, colback = blue!3,boxsep=0pt,left=0.2em, coltitle = blue!20!black, title = \textbf{jawab}, #1, } 

%---------- kunci (jika 1 ) muncul
\def\tampilkunci{1}
\newcommand{\hide}[1]{\ifnum\tampilkunci=1
%
\begin{mybox}
 #1
\end{mybox}
%
\vspace{\baselineskip}\fi}



\newcommand*\cicled[1]{\tikz[baseline=(char.base)]{\node[white, shape=circle, fill=red!80,draw,inner sep=0.5pt](char){#1};}}

\newcommand*\kunci[1]{\ifnum\tampilkunci=1
%
\tikz[baseline=(char.base)]{\node[red, shape=circle,draw,inner sep=0.5pt,xshift=2pt](char){#1};}\stepcounter{enumii}
\fi\ifnum\tampilkunci=0
%
\hspace{3pt}#1\stepcounter{enumii}
%
\fi}

\newcommand*\silang[1]{\tikz[baseline=(char.base)]{
\draw[red,thick](-0.2,-0.20)--(0.2,0.2);
\draw[red,thick](-0.2,0.20)--(0.2,-0.2);
\node[black](char){#1};
}}

\newcommand*\centang[1]{\tikz[baseline=(char.base)]{
\draw[red, very thick](-0.2,0.1)--(-0.1,0)--(0.2,0.3);
\node(char){#1};
}}

\newcommand*\merah[1]{
\textcolor{red}{#1}}
\newcommand*\pilgan[1]{
\begin{enumerate}[label=\Alph*., itemsep=0pt,topsep=0pt,leftmargin=*,align=Center] #1 
\end{enumerate}}
\newcommand*\pernyataan[1]{
\begin{enumerate}[label=(\arabic*), itemsep=0pt,topsep=0pt,leftmargin=*] #1 
\end{enumerate}}

\newcommand{\pilgani}[1]{                            \vspace{-0.3cm}\begin{multicols}{2}
 \begin{enumerate}[label=\Alph*., itemsep=0pt,topsep=0pt,leftmargin=*,align=Center]#1                     \end{enumerate}
 \phantom{ini cuma sapi, wedus, dan ayam}
 \end{multicols}}


\begin{document}


 \textbf{Kuis Usaha dan Energi} \phantom{ini nama siswa yang aaamengerjakan soal kuis ini }  

No callculator allowed !  

\begin{enumerate}

\item Bola 4 kg dilempar ke atas dari atas tanah dengan kelajuan 10 m/s. Maka energi kinetik bola saat berada pada ketinggian 2 m adalah . . . 
    \pilgan{
        \item 80 J
        \item 100 J
        \item 120 J
        \item 140 J
        \item 160 J } \vspace{2cm}
        
\item Sebuah benda dengan massa 2 kg mula-mula diam. Jika ditarik dengan gaya 4 N ke samping, maka usaha untuk memindahkan benda selama 3 sekon dan besar perpindahannya adalah . . . 
    \pilgani{
        \item 20 J dan 5 m
        \item \kunci{B.}36 J dan 9 m
        \item 40 J dan 10 m
        \item 60 J dan 15 m
        \item 100 J dan 25 m
    } \vspace{2cm}


\item Mobil dengan massa 2 ton mula2 diam. Pada saat $t$ mobil tersebut bergerak dengan energi 4 kJ. Jika mobil tersebut ditarik dengan gaya 400 N, maka waktu $t$ adalah . . . 
    \pilgani{
        \item 2 s
        \item 4 s
        \item 6 s
        \item 8 s
        \item 10 s
    }  \vspace{2cm}

\item Balok berada pada alas bawah suatu bidang miring. Balok tersebut bermassa 2 kg dan berada pada kecepatan 20 m/s. Bidang miring tersebut membentuk sudut 30$^o$. Jika balok sudah meluncur sejauh 10 m, maka perbandingan energi kinetik dan energi potensial di titik tersebut adalah . . . .
    \pilgani{
        \item 1 : 2
        \item 2 : 1
        \item 3 : 1 
        \item 2 : 3
        \item 1 : 1
    }  \vspace{2cm}

         
\item Suatu pegas ditarik dengan gaya 50 N bertambah panjang 2cm. Jika pegas tersebut digunakan untuk melemparkan anak panah sebesar 10 gram, dan ditarik sejauh 10 cm, maka kecepatan yang dihasilkan adalah . . . 
    \pilgani{
        \item 10 m/s
        \item 20 m/s
        \item 30 m/s
        \item 40 m/s
        \item 50 m/s
    }  \vspace{2cm}

\item Suatu gaya $\vec{F}=(2\hat{i}+4\hat{j})$ N digunakan untuk menggerakan benda. Benda tersebut berpindah sejauh 10 m ke arah mendatar. Maka usaha yang dihasilkan adalah . . . .
    \pilgani{
        \item 1 J
        \item 2 J
        \item 10 J
        \item 15 J
        \item 20 J 
    }  \vspace{2cm}


\item Sebuah benda didorong dengna gaya 5 N dengan gaya yang membentuk sudut $\theta$ (sin $\theta$ = 3/5). Jika massa benda adalah 5 kg, tentukan usaha setelah mendorong selama 5 sekon . . . .
	\pilgani{
	\item 20 J
	\item 30 J
	\item 40 J
	\item 50 J
	\item 60 J }   \vspace{2cm}
	
\item Sebuah balok bermassa 2 kg menumbuk pegas mendatar yang memiliki tetapan gaya 800 N/m. Balok menekan pegas sejauh 6 cm dari posisi awalnya. Bila lantai licin, kelajuan balok saat menumbuk pegas adalah . . . 
\pilgani{
	\item 4 m/s
	\item 5 m/s
	\item 6 m/s
	\item 10 m/s
	\item 12 m/s
	}    \vspace{2cm}
	
\item Suatu benda berada di puncak bidang miring dengan ketinggian $h$. Sesaat kemudian benda dilepaskan sehingga menuruni bidang miring. Pada titik A, ketinggian benda adalah $\frac{1}{5} h$. Perbandingan energi potensial dan energi kinetik pada titik tersebut adalah . . .
\pilgani{
	\item 1 : 5
	\item 5 : 1
	\item 1 : 6
	\item 6 : 1
	\item 2 : 3 }   \vspace{2cm}
	
\item Mobil dengan massa 2000 kg bergerak menanjak pada suatu bukit. Panjang lintasan dari A ke B adalah 40 m. Kelajuan awal di A sama dengan 20 m/s dan kelajuan di B sama dengan 5 m/s. Berapa gaya gesek yang dikerjakan permukaan jalan pada ban mobil selama geraknya ? .  . . . N
\pilgani{
	\item 2573
	\item 5375
	\item 3573
	\item 1300
	\item 8799
	} \vspace{2cm}
	
\item Sebuah balok licin meluncur dari tepi jurang dengan kecepatan 20 m/s. Ketinggian jurang adalah 60 m. Maka kelajuan balok saat hampir mengenai tanah adalah . . . . 
\pilgani{
	\item 20 m/s
	\item 20$\sqrt{3}$ m/s
	\item 30 m/s
	\item 40 m/s
	\item 40$\sqrt{2}$ m/s
	}\vspace{2cm}
	
\item Sebuah mesin dengan daya 12 kw digunakan untuk menaikkan elevator bermassa 900 kg dari lantai dasar sampai lantai 8. Jika lantai dasar hingga lantai 8 adalah 40 m, maka waktu yang dibutuhkan adalah . . . 
\pilgani{
	\item 20 s
	\item 7,5 s
	\item 13,3 s
	\item 35 s
	\item 3 s
	} \vspace{2cm}
	
	
	
\item Mobil bermassa 400 kg dengan kelajuan 36 km/s. Suatu saat mesin dimatikan sehingga berhenti karena gaya gesek pada mobil. Jika gaya gesek tersebut adalah 100 N, maka jarak terjauh yang dapat ditempuh mobil sebelum berhenti adalah . . . .
\pilgani{
	\item 50 m
	\item 100 m
	\item 150 m
	\item 200 m
	\item 250 m } \vspace{2cm}
	

\item Sebuah peti ( 6kg) diterik dengan gaya 50 N ke arah 37$^o$ terhadap horisontal. Sebuah gaya P menghambat gerakkan sebesar 10 N. Maka usaha total pada peti setelah bergerak sejauh 3 m adalah . . . 
\pilgani{
	\item 150 J
	\item 120 J
	\item 90 J
	\item 80 J
	\item 40 J } \vspace{2cm}
	
\item Mobil dengan massa 1 ton melaju dengan kecepatan 36 km/jam menjadi 72 km/jam dalam waktu 10 s. Maka daya keluaran mesin adalah . . .
\pilgani{
	\item 15 kW
	\item 20 kW
	\item 25 kW
	\item 30 kW
	\item 45 kW } \vspace{2cm}
	



\end{enumerate}

\end{document}







\documentclass[14pt,a4paper]{extarticle}
\usepackage[latin1]{inputenc}
\usepackage{amsmath}
\usepackage{microtype}
\usepackage[none]{hyphenat}
\usepackage{verbatim}
\usepackage{amsfonts}
\usepackage{amssymb}
\usepackage{enumitem}
\renewcommand{\familydefault}{\sfdefault}
\usepackage{mathpazo}
\renewcommand{\rmdefault}{put}
\usepackage{enumitem}
\usepackage[dvipsnames,svgnames]{xcolor}
\usepackage{tkz-euclide}
\usetkzobj{all}
\usepackage{graphicx}
\usepackage{tikz} 	
\usepackage{adjustbox}
\usepackage{multicol}
\usepackage{lipsum}
\usepackage[left=0.7cm,right=1cm,top=1cm,bottom=1.5cm]{geometry}
\usepackage{cancel} \usepackage{xcolor}
\usepackage{tcolorbox}
\usetikzlibrary{decorations.pathmorphing,patterns}
\usetikzlibrary{decorations.pathreplacing,calc}
 \newcommand\coret[2][red]{\renewcommand\CancelColor{\color{#1}}\cancel{#2}}
\SetLabelAlign{Center}{\hfil\makebox[1.0em]{#1}\hfil}

%%_------= solusi


% Set this =0 to hide, =1 to show

% Set this =0 to hide, =1 to show
\newtcolorbox{mybox}[1][] { colframe = blue!10, colback = blue!3,boxsep=0pt,left=0.2em, coltitle = blue!20!black, title = \textbf{jawab}, #1, } 

%---------- kunci (jika 1 ) muncul
\def\tampilkunci{1}
\newcommand{\hide}[1]{\ifnum\tampilkunci=1
%
\begin{mybox}
 #1
\end{mybox}
%
\vspace{\baselineskip}\fi}



\newcommand*\cicled[1]{\tikz[baseline=(char.base)]{\node[white, shape=circle, fill=red!80,draw,inner sep=0.5pt](char){#1};}}

\newcommand*\kunci[1]{\ifnum\tampilkunci=1
%
\tikz[baseline=(char.base)]{\node[red, shape=circle,draw,inner sep=0.5pt,xshift=2pt](char){#1};}\stepcounter{enumii}
\fi\ifnum\tampilkunci=0
%
\hspace{3pt}#1\stepcounter{enumii}
%
\fi}

\newcommand*\silang[1]{\tikz[baseline=(char.base)]{
\draw[red,thick](-0.2,-0.20)--(0.2,0.2);
\draw[red,thick](-0.2,0.20)--(0.2,-0.2);
\node[black](char){#1};
}}

\newcommand*\centang[1]{\tikz[baseline=(char.base)]{
\draw[red, very thick](-0.2,0.1)--(-0.1,0)--(0.2,0.3);
\node(char){#1};
}}

\newcommand*\merah[1]{
\textcolor{red}{#1}}
\newcommand*\pilgan[1]{
\begin{enumerate}[label=\Alph*., itemsep=0pt,topsep=0pt,leftmargin=*,align=Center] #1 
\end{enumerate}}
\newcommand*\pernyataan[1]{
\begin{enumerate}[label=(\arabic*), itemsep=0pt,topsep=0pt,leftmargin=*] #1 
\end{enumerate}}

\newcommand{\pilgani}[1]{                            \vspace{-0.3cm}\begin{multicols}{2}
 \begin{enumerate}[label=\Alph*., itemsep=0pt,topsep=0pt,leftmargin=*,align=Center]#1                     \end{enumerate}
 \phantom{ini cuma sapi, wedus, dan ayam}
 \end{multicols}}


\begin{document}


 \textbf{Kuis Gravitasi} \phantom{ini nama siswa yang aaamengerjakan soal kuis ini }  

No callculator allowed !  
$$ G = 6,7 \times 10^{-11}$$

\begin{enumerate}

\item Berat di bumi adalah 900N. Berat benda tersebut jika berada pada ketinggian 2R adalah. . . . \pilgani{
    \item 225 N
    \item 450 N
    \item 1800 N
    \item 180 N
    \item 100 N
}

\item Berat seorang astronot di Bumi adalah 800 N. Astronot bepergian ke planet X yang mempunyai massa 16 kali bumi dan jari-jari 2 kali bumi. Maka berat astronot tersebut saat berada di ketinggian 3R dari permukaan planet X adalah . . . .
\pilgani {
    \item 3200 N
    \item 3200/9 N
    \item 800 N
    \item 800/3 N
    \item 200 N }

\item Suatu planet mempunyai massa 4 kali bumi dan jari-jari 3 kali bumi. Maka percepatan gravitasi di planet tersebut adalah . . . 
\pilgani{
    \item 2$g$
    \item $\frac{4}{3} g$
    \item $\frac{3}{4} g$
    \item $\frac{4}{9} g$
    \item $\frac{9}{4} g$
    }

\item Seorang peneliti berada di planet yang berjari-jari 1000km. Jika percepatan gravitasi di planet adalah 8 m/s$^2$,maka kecepatan minimum untuk lepas dari planet adalah . . . 
\pilgani{
    \item 2 km/s
    \item $\sqrt{8}$ km/s
    \item 4 km/s
    \item 4$\sqrt{10}$km/s
    \item 16 km/s
}

\item Tiga buah benda masing-masing 1kg, jika jarak A dan B 1m, B dan C 1 m, dan B ada di siku-siku. Maka besar gaya di C  adalah . , , , 
\pilgani{
    \item $\sqrt{2}$ G
    \item $\sqrt{2+\sqrt{2}}$ G
    \item $\sqrt{3}$ G
    \item $ 2 \sqrt{2}$ G
    \item $\frac{1}{2} \sqrt{5+2\sqrt{2}}$ G
}



\item Dua massa masing-masing 20 kg berada pada jarak 8 m. Gaya tarik kedua massa tersebut adalah . . . 
\pilgani{
    \item 8,32 $\times 10^{-10}$ 
    \item 6,24 $\times 10^{-10}$  
    \item 4,16 $\times 10^{-10}$ 
    \item 2,08 $\times 10^{-10}$ 
    \item 1,04  $\times 10^{-10}$ 
    } 

\end{enumerate}

\end{document}







\documentclass[14pt,a4paper]{extarticle}
\usepackage[latin1]{inputenc}
\usepackage{amsmath}
\usepackage{microtype}
\usepackage[none]{hyphenat}
\usepackage{verbatim}
\usepackage{amsfonts}
\usepackage{amssymb}
\usepackage{enumitem}
\renewcommand{\familydefault}{\sfdefault}
\usepackage{mathpazo}
\renewcommand{\rmdefault}{put}
\usepackage{enumitem}
\usepackage[dvipsnames,svgnames]{xcolor}
\usepackage{tkz-euclide}
\usetkzobj{all}
\usepackage{graphicx}
\usepackage{tikz} 	
\usepackage{adjustbox}
\usepackage{multicol}
\usepackage{lipsum}
\usepackage[left=0.7cm,right=1cm,top=1cm,bottom=1.5cm]{geometry}
\usepackage{cancel} \usepackage{xcolor}
\usepackage{tcolorbox}
\usetikzlibrary{decorations.pathmorphing,patterns}
\usetikzlibrary{decorations.pathreplacing,calc}
 \newcommand\coret[2][red]{\renewcommand\CancelColor{\color{#1}}\cancel{#2}}
\SetLabelAlign{Center}{\hfil\makebox[1.0em]{#1}\hfil}

%%_------= solusi


% Set this =0 to hide, =1 to show

% Set this =0 to hide, =1 to show
\newtcolorbox{mybox}[1][] { colframe = blue!10, colback = blue!3,boxsep=0pt,left=0.2em, coltitle = blue!20!black, title = \textbf{jawab}, #1, } 

%---------- kunci (jika 1 ) muncul
\def\tampilkunci{1}
\newcommand{\hide}[1]{\ifnum\tampilkunci=1
%
\begin{mybox}
 #1
\end{mybox}
%
\vspace{\baselineskip}\fi\ifnum\tampilkunci=0
%
\vspace{2cm}
%
\fi}



\newcommand*\cicled[1]{\tikz[baseline=(char.base)]{\node[white, shape=circle, fill=red!80,draw,inner sep=0.5pt](char){#1};}}

\newcommand*\kunci[1]{\ifnum\tampilkunci=1
%
\tikz[baseline=(char.base)]{\node[red, shape=circle,draw,inner sep=0.5pt,xshift=2pt](char){#1};}\stepcounter{enumii}
\fi\ifnum\tampilkunci=0
%
\hspace{3pt}#1\stepcounter{enumii}
%
\fi}

\newcommand*\silang[1]{\tikz[baseline=(char.base)]{
\draw[red,thick](-0.2,-0.20)--(0.2,0.2);
\draw[red,thick](-0.2,0.20)--(0.2,-0.2);
\node[black](char){#1};
}}

\newcommand*\centang[1]{\tikz[baseline=(char.base)]{
\draw[red, very thick](-0.2,0.1)--(-0.1,0)--(0.2,0.3);
\node(char){#1};
}}

\newcommand*\merah[1]{
\textcolor{red}{#1}}
\newcommand*\pilgan[1]{
\begin{enumerate}[label=\Alph*., itemsep=0pt,topsep=0pt,leftmargin=*,align=Center] #1 
\end{enumerate}}
\newcommand*\pernyataan[1]{
\begin{enumerate}[label=(\arabic*), itemsep=0pt,topsep=0pt,leftmargin=*] #1 
\end{enumerate}}

\newcommand{\pilgani}[1]{                            \vspace{-0.3cm}\begin{multicols}{2}
 \begin{enumerate}[label=\Alph*., itemsep=0pt,topsep=0pt,leftmargin=*,align=Center]#1                     \end{enumerate}
 \phantom{ini cuma sapi, wedus, dan ayam}
 \end{multicols}}


\begin{document}


 \textbf{Usaha dan Energi} \phantom{ini nama siswa yang aaamengerjakan soal kuis ini }  

No callculator allowed !  

\begin{enumerate}

\item Bola 4 kg dilempar ke atas dari atas tanah dengan kelajuan 10 m/s. Maka energi kinetik bola saat berada pada ketinggian 2 m adalah . . . 
    \pilgan{
        \item 80 J
        \item 100 J
        \item[\kunci{C.}] 120 J
        \item 140 J
        \item 160 J }
        
        \hide{
        Karena diketahui kecepatan/kelajuan lalau ditanyakan energi kinetik pada ketinggian (ada EP dan EK) maka gunakan kekekalan
        \begin{align*}
        EM_1 &= EM_2 \\
        \frac{1}{2}mv^2 + mgh_1 &= EP_2 + EK_2\\
        \frac{1}{2}4(10)^2 + 4.10.0 & = 4.10.2 + EK_2 \\
        200 & = 80 + EK_2 \\
        120 &= EK_2
        \end{align*}
                 }
        
        
\item Sebuah benda dengan massa 2 kg mula-mula diam. Jika ditarik dengan gaya 4 N ke samping, maka usaha untuk memindahkan benda selama 3 sekon dan besar perpindahannya adalah . . . 
    \pilgani{
        \item 20 J dan 5 m
      \item[\kunci{B.}] 36 J dan 9 m
        \item 40 J dan 10 m
        \item 60 J dan 15 m
        \item 100 J dan 25 m     } 
        
        \hide{
        Usaha bisa dikerjakan menggunakan $W= F.s$ atau $W = \Delta EK = \Delta EP$. Pada soal ini ditarik ke samping, berarti tidak terjadi perubahan ketinggian (energi potensial). Gunakan saja perubahan energi kinetik
        \begin{align*}
        W &= \Delta EK \\
        W &= EK_2 - EK_1\\
        W &= \frac{1}{2} 2 v^2 - 0\\
        \end{align*} cari dulu kecepatannya pakai persamaan GLBB
        \begin{align*}
        v&=v_o + at\\
        v&=0+\frac{F}{m}3 \\
        v&=\frac{4}{2}3=6\\
        W & = \frac{1}{2} 2 . 6^2 = 36 \text{J}\\
        W &= F.s \\
        36 &=4.s\\
        s&=9 \text{m}
        \end{align*}}


\item Mobil dengan massa 2 ton mula2 diam. Pada saat $t$ mobil tersebut bergerak dengan energi kinetik 4 kJ. Jika mobil tersebut ditarik dengan gaya 400 N, maka waktu $t$ adalah . . . 
    \pilgani{
        \item 2 s
        \item 4 s
        \item 6 s
        \item 8 s
        \item[\kunci{E.}] 10 s
    } 
    \hide{
    \begin{align*}
    EK&=\frac{1}{2} 2000 v^2  \\
    4000&=1000.v^2\\
    v&=2 \text{m/s}\\
    \phantom{ as }\\
    v&=v_o+at\\
    2&=0+\frac{F}{m}t\\
    2&=0+\frac{400}{2000}t\\
    t&=10\text{s}
    \end{align*}
    
    }

\item Balok berada pada alas bawah suatu bidang miring. Balok tersebut bermassa 2 kg dan berada pada kecepatan 20 m/s. Bidang miring tersebut membentuk sudut 30$^o$. Jika balok sudah naik bidang miring sejauh 10 m, maka perbandingan energi kinetik dan energi potensial di titik tersebut adalah . . . .
    \pilgani{
        \item 1 : 2
        \item 2 : 1
        \item[\kunci{C.}] 3 : 1 
        \item 2 : 3
        \item 1 : 1     } 
        
\hide{
    Pada saat di bawah, balok punya kecepatan tapi tidak punya energi potensial. Saat dia meluncur sejauh 10 m pada bidang miring, maka ketinggiannya menjadi 5 m (silakan gambar, tinggi adalah sinus). Pertanyaan adalah energi kinetik dan potensial. maka gunakan $EM_1 = EM_2 $
    \begin{align*}
    EM_1 &= EM_2 \\
    \frac{1}{2}mv^2 + mg.0 &= EK + EP \\
    \frac{1}{2}m.400 &= EK + m.g.5\\
    200m &= EK + 50m \\
    EK &= 150m\\
    EK : EP &= 150 : 50 =  3 : 1 \\
    \end{align*}
}
    
         
\item Suatu pegas ditarik dengan gaya 50 N bertambah panjang 2cm. Jika pegas tersebut digunakan untuk melemparkan anak panah sebesar 10 gram, dan ditarik sejauh 10 cm, maka kecepatan yang dihasilkan adalah . . . 
    \pilgani{
        \item 10 m/s
        \item 20 m/s
        \item 30 m/s
        \item 40 m/s
        \item[\kunci{E.}] 50 m/s     }
\hide{ 
    Pada kasus ini, energi potensial berubah menjadi energi kinetik
    \begin{align*}
    EP&=EK\\
    \frac{1}{2}k(\Delta x)^2 &= \frac{1}{2}mv^2\\
    \phantom{aasss}\\
    F&=k.\Delta x\\
    \frac{F}{\Delta x } &= k\\
    \frac{50}{0,02} &=k\\
    k&=2500 \text{  N/m}\\
    \phantom{ssdd}\\
    \frac{1}{2}2500(\Delta x)^2 &= \frac{1}{2}0,01v^2\\
    2500 &= v^2\\
    v&=50\text { m/s}
    \end{align*}

}


\item Suatu gaya $\vec{F}=(2\hat{i}+4\hat{j})$ N digunakan untuk menggerakan benda. Benda tersebut berpindah sejauh 10 m ke arah mendatar. Maka usaha yang dihasilkan adalah . . . .
    \pilgani{
        \item 1 J
        \item 2 J
        \item 10 J
        \item 15 J
        \item[\kunci{E.}] 20 J     }  

        \hide{
        Karena benda mendatar, berarti bergerak dengan persamaan perpindahan $\vec{r}= 10\hat{i}$
        \begin{align*}
        W &= \vec{F}\bullet \vec{s}\\
        W &= (2\hat{i}+4\hat{j})\bullet (10\hat{i})\\
        W &= 20 \text{ J}
        \end{align*}

        }


\item Sebuah benda didorong dengna gaya 5 N dengan gaya yang membentuk sudut $\theta$ (sin $\theta$ = 3/5). Jika massa benda adalah 5 kg, tentukan usaha setelah mendorong selama 5 sekon . . . .
	\pilgani{
	\item 20 J
	\item 30 J
	\item[\kunci{C.}] 40 J
	\item 50 J
	\item 60 J }   
\hide{
Usaha pada soal ini adalah perubahan kecepatan yakni perubahan energi kinetik. 
\begin{align*}
    W &= \Delta EK = EK_2-EK_1\\
    W &= \frac{1}{2}mv^2 - 0\\
    W &= \frac{1}{2}5.v^2\\
    \phantom{a}
    v&=v_o + at\\
    v&=\frac{F}{m}t\\
    v&=\frac{(5 \cos \theta)}{5}5\\
    v&=\frac{4}{5}5=4 \text{ m/s}\\
    \phantom{b}
    W &=\frac{1}{2}5.4^2 = 40 \text{ J}
\end{align*}

}
	
\item Sebuah balok bermassa 2 kg menumbuk pegas mendatar yang memiliki tetapan gaya 800 N/m. Balok menekan pegas sejauh 6 cm dari posisi awalnya. Bila lantai licin, kelajuan balok saat menumbuk pegas adalah . . . 
\pilgani{
	\item 4 cm/s
	\item 5 cm/s
	\item 6 cm/s
	\item 10 cm/s
	\item[\kunci{E.}]12 cm/s	} 				 \hide{
 Balok menmbuk pegas. Berarti balok awalnya mempunyai kecepatan, kemudian berhenti. Energi kinetik berubah menjadi energi potensial pegas.
 \begin{align*}
 EK &= EP_{\text{pegas}}\\
 \frac{1}{2} mv^2 &= \frac{1}{2} k (\Delta x )^2\\
 \frac{1}{2}2v^2&=\frac{1}{2}800(0,06)^2\\
 v^2&=\frac{144}{10000}=0,12 \text{ m/s}
 \end{align*}
    }
	
\item Suatu benda berada di puncak bidang miring dengan ketinggian $h$. Sesaat kemudian benda dilepaskan sehingga menuruni bidang miring. Pada titik A, ketinggian benda adalah $\frac{1}{5} h$. Perbandingan energi potensial dan energi kinetik pada titik tersebut adalah . . .
\pilgani{
	\item 1 : 5
	\item 5 : 1
	\item [\kunci{C.}]1 : 4
	\item 4 : 1
	\item 2 : 3 }  
    
\hide{
Ketinggian berkurang, ditanyakan EP dan EK. Pasti digunakan $EM_1 = EM_2 $
\begin{align*}
EM_1 &= EM_2\\
mgh &= mg(\frac{1}{5}h)+ EK_2\\
mgh &= \frac{1}{5} mgh + EK_2\\
mgh &= \frac{1}{5} mgh + \frac{4}{5}mgh\\
EP_2 : EK_2 &= 1 : 4 
\end{align*}    
}
	
\item Mobil dengan massa 2000 kg bergerak pada suatu aspal. Panjang lintasan dari A ke B adalah 40 m. Kelajuan awal di A sama dengan 20 m/s dan kelajuan di B sama dengan 5 m/s. Berapa gaya gesek yang dikerjakan permukaan jalan pada ban mobil selama geraknya ? .  . . . N
\pilgani{
	\item 2573
	\item[\kunci{B.}] 9375
	\item 3573
	\item 1300
	\item 8799
	} 
\hide{
Gaya gesek ada hubungannya dengan usaha. Padahal usaha juga ada hubungannya dengan perubahan energi kinetik
\begin{align*}
W &= \Delta EK\\
F.s &= EK_2-EK_1\\
F.40 &= \frac{1}{2}m(v_2^2-v_1^1)\\
F.40 &= \frac{1}{2}2000(400-25)\\
F&=9375 \text { N}
\end{align*}
}
	
\item Sebuah balok licin meluncur dari tepi jurang dengan kecepatan 20 m/s. Ketinggian jurang adalah 60 m. Maka kelajuan balok saat hampir mengenai tanah adalah . . . . 
\pilgani{
	\item 20 m/s
	\item 20$\sqrt{3}$ m/s
	\item 30 m/s
	\item[\kunci{D.}] 40 m/s
	\item 40$\sqrt{2}$ m/s
	}
\hide{
 Balok awalnya pada ketnggian 60 m, dan bergerak dengan kecepatan 20 m/s. Berarti pada keadaan awal balok punya EP dan EK. Lalu menuruni jurang. Di dasar jurang balok tidak ada ketinggian, hanya kecepatan. Maka digunakan kekekalan energi
 \begin{align*}
 EM_1 &= EM_2\\
 EP_1 + EK_1 &= EK_2 + EP_2\\
 m.g.h + \frac{1}{2}mv_1^2 &= \frac{1}{2}mv_2^2+0\\
 10.60 + \frac{1}{2}20^2&=\frac{1}{2}v_2^2\\
 v_2^2 &= 1600 \\
 v_2 &= 40\text{ m/s}
 \end{align*}}
	
\item Sebuah mesin dengan daya 12 kw digunakan untuk menaikkan elevator bermassa 900 kg dari lantai dasar sampai lantai 8. Jika lantai dasar hingga lantai 8 adalah 40 m, maka waktu yang dibutuhkan adalah . . . 
\pilgani{
	\item 20 s
	\item 7,5 s
	\item 13,3 s
	\item 35 s
	\item[\kunci{E.}] 3 s
	}
    
 \hide{
 \begin{align*}
 P &= \frac{W}{t}\\
 t &= \frac{W}{P}\\
 t &= \frac{\Delta EP}{P}\\
 t &= \frac{900.10.40}{12000}\\
 t &= 12\text{ s}
 \end{align*}
 }
	
\item Mobil bermassa 400 kg dengan kelajuan 36 km/s. Suatu saat mesin dimatikan sehingga berhenti karena gaya gesek pada mobil. Jika gaya gesek tersebut adalah 100 N, maka jarak terjauh yang dapat ditempuh mobil sebelum berhenti adalah . . . .
\pilgani{
	\item 50 m
	\item 100 m
	\item 150 m
	\item[\kunci{D.}] 200 m
	\item 250 m }
\hide{
   \begin{align*}
   W &= \Delta EK\\
   F.s &= EK_2 - EK_1\\
   100.s &= 0 - \frac{1}{2}400.10^2\\
   s&=200\text{ m}
   \end{align*}
}
  

\item Sebuah peti ( 6kg) diterik dengan gaya 50 N ke arah 37$^o$ terhadap horisontal. Sebuah gaya P menghambat gerakkan sebesar 10 N. Maka usaha total pada peti setelah bergerak sejauh 3 m adalah . . . 
\pilgani{
	\item 150 J
	\item 120 J
	\item[\kunci{C.}] 90 J
	\item 80 J
	\item 40 J }
\hide{
    Hati-hati pada soal ini,karena gaya ada dua. Gaya yang pertama ada sudutnya. Sehingga yang terpakai adalah $F\cos 37^o$ dan gaya yang menahan yakni 10 N
    \begin{align*}
    W &= F.s\\
    W &= (40-30).3\\
    W &=90 \text{ N}
    \end{align*}}
	
\item Mobil dengan massa 1 ton melaju dengan kecepatan 36 km/jam menjadi 72 km/jam dalam waktu 10 s. Maka daya keluaran mesin adalah . . .
\pilgani{
	\item[\kunci{A.}] 15 kW
	\item 20 kW
	\item 25 kW
	\item 30 kW
	\item 45 kW } 
	
\hide{
    \begin{align*}
    W &= \Delta EK = EK_2 - EK_1\\
    W &= \frac{1}{2}m(v_2^2-v_1^2)\\
    W &= \frac{1}{2}1000(400-100)\\
    W &= 150000\text{ J}\\
    \phantom{2}
    P&=\frac{W}{t}=\frac{150000}{10}=15 \text{ kW}
    \end{align*}}

\end{enumerate}

\end{document}







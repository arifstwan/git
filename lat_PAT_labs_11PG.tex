\documentclass[10pt,a4paper]{extarticle}
\usepackage[latin1]{inputenc}
\usepackage{amsmath}
\usepackage{microtype}
\usepackage[none]{hyphenat}
\usepackage{verbatim}
\usepackage{amsfonts}
\usepackage{amssymb}
\usepackage{enumitem}
\renewcommand{\familydefault}{\sfdefault}
\usepackage{mathpazo}
\renewcommand{\rmdefault}{put}
\usepackage{enumitem}
\usepackage[dvipsnames,svgnames]{xcolor}
\usepackage{tkz-euclide}
\usetkzobj{all}
\usepackage{graphicx}
\usepackage{tikz} 	
\usepackage{adjustbox}
\usepackage{multicol}
\usepackage{lipsum}
\usepackage[left=0.7cm,right=1cm,top=1cm,bottom=1.5cm]{geometry}
\usepackage{cancel} \usepackage{xcolor}
\usepackage{tcolorbox}
\usetikzlibrary{decorations.pathmorphing,patterns}
\usetikzlibrary{decorations.pathreplacing,calc}
 \newcommand\coret[2][red]{\renewcommand\CancelColor{\color{#1}}\cancel{#2}}
\SetLabelAlign{Center}{\hfil\makebox[1.0em]{#1}\hfil}

%%_------= solusi


% Set this =0 to hide, =1 to show

% Set this =0 to hide, =1 to show
\newtcolorbox{mybox}[1][] { colframe = blue!10, colback = blue!3,boxsep=0pt,left=0.2em, coltitle = blue!20!black, title = \textbf{jawab}, #1, } 

%---------- kunci (jika 1 ) muncul
\def\tampilkunci{1}
\newcommand{\hide}[1]{\ifnum\tampilkunci=1
%
\begin{mybox}
 #1
\end{mybox}
%
\vspace{\baselineskip}\fi}



\newcommand*\cicled[1]{\tikz[baseline=(char.base)]{\node[white, shape=circle, fill=red!80,draw,inner sep=0.5pt](char){#1};}}

\newcommand*\kunci[1]{\ifnum\tampilkunci=1
%
\tikz[baseline=(char.base)]{\node[red, shape=circle,draw,inner sep=0.5pt,xshift=2pt](char){#1};}\stepcounter{enumii}
\fi\ifnum\tampilkunci=0
%
\hspace{3pt}#1\stepcounter{enumii}
%
\fi}

\newcommand*\silang[1]{\tikz[baseline=(char.base)]{
\draw[red,thick](-0.2,-0.20)--(0.2,0.2);
\draw[red,thick](-0.2,0.20)--(0.2,-0.2);
\node[black](char){#1};
}}

\newcommand*\centang[1]{\tikz[baseline=(char.base)]{
\draw[red, very thick](-0.2,0.1)--(-0.1,0)--(0.2,0.3);
\node(char){#1};
}}

\newcommand*\merah[1]{
\textcolor{red}{#1}}
\newcommand*\pilgan[1]{
\begin{enumerate}[label=\Alph*., itemsep=0pt,topsep=0pt,leftmargin=*,align=Center] #1 
\end{enumerate}}
\newcommand*\pernyataan[1]{
\begin{enumerate}[label=(\arabic*), itemsep=0pt,topsep=0pt,leftmargin=*] #1 
\end{enumerate}}

\newcommand{\pilgani}[1]{                            \vspace{-0.3cm}\begin{multicols}{2}
 \begin{enumerate}[label=\Alph*., itemsep=0pt,topsep=0pt,leftmargin=*,align=Center]#1                     \end{enumerate}
 \phantom{ini cuma sapi, wedus, dan ayam}
 \end{multicols}}


\begin{document}


 \textbf{Latihan PAT Effek Rumah Kaca} \phantom{ini nama siswa yang aaamengerjakan soal kuis ini }  

\begin{multicols*}{3}\raggedcolumns

\begin{enumerate}
\item Radiasi panas matahari yang terkurung dalam atmosfir bumi, serta meningkatnya panas oleh pengikatan CO$_2$ dikenal sebagai . . . 
\pilgan{
\item pemanasan global
\item gas rumah kaca
\item efek rumah kaca
\item polusi suara
\item daya lenting lingkungan}

\item Berikut ini yang \textit{bukan} termasuk akibat dari ``efek rumah kaca'' adalah
\pilgan{
\item berkurangnya areal hutan
\item meningkatnya kematian manusia karena penyakit
\item naiknya suhu bumi
\item turunnya permukaan air laut
\item mencairnya es di daerah kutub}

\item Gas yang timbul dari tempat pembuangan sampah adalah 
\pilgani{
\item SO$_2$
\item O$_3$
\item H$_2$S
\item CO$_2$
\item CH$_4$
}

\item Mekanisme efek rumah kaca yang normal sebenarnya sangat diperlukan bagi kehidupan karena . .
\pilgan{
\item mencegah lubang ozon
\item mengurangi polusi udara
\item menghambat radiasi untuk atmosfer bumi
\item menghangatkan suhu bumi sehingga nyaman untuk ditinggali
\item menyerap gas rumah kaca sehingga tidak terjadi pemanasan berlebihan }

\item Penggunaan CFC pada berbagai produk dibatasi karena
\pilgani{
\item kanker
\item hujan asam
\item keracunan
\item lubang ozon
\item asap kabut}


\item Salah satu dampak pemanasan global adalah
\pilgan{
\item penurunan permukaan laut
\item timbul keracunan CO
\item terjadi gempa
\item rusaknya bahan logam karena korosi
\item timbul penyakit}

\item Penggunaan pendingin (AC, lemari es) berdampak
\pilgan{
\item menipisnya lapisan ozon
\item gangguan pernapasan
\item menipisnya lapisan stratosfer
\item menipisnya atmosfer
\item timbul penyakit kulit}

\item Sumber emisi global yang menghasilkan gas karbon dioksida terbesar adalah
\pilgan{
\item kebakaran hutan
\item pembakaran batu bara
\item penggunaan gas alam
\item kendaraan bermotor
\item kilang minyak}

\item Keuntungan penghijauan antara lain karena tanaman dapat . . .
\pilgan{
\item mengikatkan gas N$_2$
\item menjaga keseimbangan CO$_2$, N$_2$, dan O$_2$
\item mengikat CO$_2$ di udara dan membebaskan O$_2$
\item mengubah CO$_2$ menjadi O$_2$
\item menyerap limbah-limbah industri }


\item Berikut ini yang tergolong gas rumah kaca adalah . . 
\pilgan{
\item CO$_2$, metana, CFC dan oksigen
\item CO$_2$, metana, CFC, dan ozon
\item CO$_2$, metanta, CFC, dan nitrogen
\item metana, CFC, O$_2$, dan uap air
\item metana, CFC, uap air dan helium
}

\item Gas yang dapat menimbulkan hujan asam adalah . . .
\pilgani {
\item SO$_2$
\item O$_3$
\item H$_2$S
\item CO$_2$
\item S$_2$ }

\item Asap kabut dapat menimbulkan kematian karena
\pilgan{
   \item menimbulkan stress
   \item menyebabkan gangguan pernapasan
   \item menimbulkan kelainan pada jantung
   \item mengganggu suplai oksigen
   \item merusak ginjal}

\item Fungsi ozon di lapisan stratosfer adalah
\pilgan{
   \item pelindung bumi dan panas matahari
   \item pelindung bumi dan sinar UV
   \item pelindung bumi dan pengaruh gerhana matahari
   \item pelindung bumi dari pengaruh bintang
   \item semua benar }

\item Meningkatnya kadar karbondioksida di udara dapat menyebabkan 
\pilgani{
   \item rusak lapisan ozon
   \item suhu udara turun
   \item korosi logam
   \item hujan asam
   \item efek rumah kaca}

\item Gas berikut yang dapat mengikat gas ozon sehingga sabuk alam ozon dapat berlubang adalah
\pilgani{
   \item CFC
   \item CH$_4$
   \item H$_2$O
   \item CO$_2$
   \item CO }

\item Ozon dapat dijumpai di atmosfer bumi pada lapisan 
\pilgan{
   \item litosfer dan stratosfer
   \item litosfer dan ionosfer
   \item stratosfer dan ionosfer
   \item stratosfer dan troposfer
   \item troposfer dan ionosfer
}

\item sinar UV matahari hanya sedikit yang sampai ke permukaan bumi karena disaring gas . . .
\pilgani{
   \item O$_3$
   \item CO
   \item NH$_3$
   \item N$_2$
   \item Ar }

\item Gas yang memiliki kemampuan seribu kali lebih efektif dalam mencegah panas keluar dari atmosfer dibanding karbon dioksida dan meyebabkan berlubangnya lapisan ozon adalah . . .
\pilgani{
   \item CO
   \item CH$_4$
   \item CFC
   \item SF$_6$
   \item NO$_2$
}
\item Dampak negatif dari membuang limbah padat sembarangan, \textit{kecuali}
\pilgan{
   \item mengurangi keindahan lingkungan 
   \item menurunkan kualitas tanah
   \item berkembangnya berbagai jenis penyakit
   \item kesuburan tanah meningkat
   \item krisis air bersih}

\item Cara yang dapat dilakuka untuk mencegah meluasnya kerusakan lapisan ozon adalah . . 
\pilgan{
   \item tidak melakukan aktivitas
   \item mengurangi penggunaan bahan ODS
   \item mengadakan penghijauan
   \item mengganti bahan bakar yang ramah lingkungan
   \item tebang pilih kayu hutan
}


\end{enumerate}
\end{multicols*}\end{document}







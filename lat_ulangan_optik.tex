\documentclass[10pt,a4paper]{extarticle}
\usepackage[latin1]{inputenc}
\usepackage{amsmath}
\usepackage{microtype}
\usepackage[none]{hyphenat}
\usepackage{verbatim}
\usepackage{amsfonts}
\usepackage{amssymb}
\usepackage{enumitem}
\renewcommand{\familydefault}{\sfdefault}
\usepackage{mathpazo}
\renewcommand{\rmdefault}{put}
\usepackage{enumitem}
\usepackage[dvipsnames,svgnames]{xcolor}
\usepackage{tkz-euclide}
\usetkzobj{all}
\usepackage{graphicx}
\usepackage{tikz} 	
\usepackage{adjustbox}
\usepackage{multicol}
\usepackage{lipsum}
\usepackage[left=0.7cm,right=1cm,top=1cm,bottom=1.5cm]{geometry}
\usepackage{cancel} \usepackage{xcolor}
\usepackage{tcolorbox}
\usetikzlibrary{decorations.pathmorphing,patterns}
\usetikzlibrary{decorations.pathreplacing,calc}
 \newcommand\coret[2][red]{\renewcommand\CancelColor{\color{#1}}\cancel{#2}}
\SetLabelAlign{Center}{\hfil\makebox[1.0em]{#1}\hfil}

%%_------= solusi


% Set this =0 to hide, =1 to show

% Set this =0 to hide, =1 to show
\newtcolorbox{mybox}[1][] { colframe = blue!10, colback = blue!3,boxsep=0pt,left=0.2em, coltitle = blue!20!black, title = \textbf{jawab}, #1, } 

%---------- kunci (jika 1 ) muncul
\def\tampilkunci{1}
\newcommand{\hide}[1]{\ifnum\tampilkunci=1
%
\begin{mybox}
 #1
\end{mybox}
%
\vspace{\baselineskip}\fi}



\newcommand*\cicled[1]{\tikz[baseline=(char.base)]{\node[white, shape=circle, fill=red!80,draw,inner sep=0.5pt](char){#1};}}

\newcommand*\kunci[1]{\ifnum\tampilkunci=1
%
\tikz[baseline=(char.base)]{\node[red, shape=circle,draw,inner sep=0.5pt,xshift=2pt](char){#1};}\stepcounter{enumii}
\fi\ifnum\tampilkunci=0
%
\hspace{3pt}#1\stepcounter{enumii}
%
\fi}

\newcommand*\silang[1]{\tikz[baseline=(char.base)]{
\draw[red,thick](-0.2,-0.20)--(0.2,0.2);
\draw[red,thick](-0.2,0.20)--(0.2,-0.2);
\node[black](char){#1};
}}

\newcommand*\centang[1]{\tikz[baseline=(char.base)]{
\draw[red, very thick](-0.2,0.1)--(-0.1,0)--(0.2,0.3);
\node(char){#1};
}}

\newcommand*\merah[1]{
\textcolor{red}{#1}}
\newcommand*\pilgan[1]{
\begin{enumerate}[label=\Alph*., itemsep=0pt,topsep=0pt,leftmargin=*,align=Center] #1 
\end{enumerate}}
\newcommand*\pernyataan[1]{
\begin{enumerate}[label=(\arabic*), itemsep=0pt,topsep=0pt,leftmargin=*] #1 
\end{enumerate}}

\newcommand{\pilgani}[1]{                            \vspace{-0.3cm}\begin{multicols}{2}
 \begin{enumerate}[label=\Alph*., itemsep=0pt,topsep=0pt,leftmargin=*,align=Center]#1                     \end{enumerate}
 \phantom{ini cuma sapi, wedus, dan ayam}
 \end{multicols}}


\begin{document}


 \textbf{Latihan Optika Geometri} \phantom{ini nama siswa yang aaamengerjakan soal kuis ini }  

\begin{multicols*}{2}

\begin{enumerate}
\item Dua buah cermin dipasang berdekatan dengan sudut 30$^o$kemudian di depannya diletakkan sebuah benda, maka banyaknya bayangan yang terbentuk adalah . . .
\pilgani{
	\item 3
	\item 6
	\item 11
	\item 20
	\item 30
}
\vspace{2cm}

\item Sebuah benda diletakkan 9 cm di depan cermin cekung yang memiliki jari-jari kelengkungan 20 cm. Sifat-sifat bayangan benda yang dihasilkan cermin adalah . . . 
\pilgani{
	\item Maya, tegak, diperbesar
	\item Maya, tegakm diperkecil
	\item Nyata, tegak, diperbesar
	\item Nyata, terbalik, diperbesar
	\item Nyata, terbalik, diperkecil
}
\vspace{3cm}

\item Dua cermin yang masing-masing panjangnya 1,6 m disusun berhadapan seperti pada gambar di bawah ini. Jarak antara cermin 20 cm. Suatu berkas sinar jatuh tepat pada salah satu ujung cermin dengan sudut 30$^o$. Sinar akan keluar dari pasangan cermin setelah mengalami pemantulan sebanyak . . . 
\pilgani{
	\item 16
	\item 13
	\item 15
	\item 9
	\item 4
}
\vspace{4cm}

\item Indeks bias udara besarnya 1, indeks bias air 4/3, dan indeks bias bahan suatu lensa tipis 3/2. Suatu lensa tipis yang kekuatannya di udara 4 dioptri di dalam air akan menjadi . . . 
\pilgani{
	\item 3/5 dioptri
	\item 1 dioptri
	\item 5/2 dioptri
	\item 5/4 dioptri
	\item 5/3 dioptri
}
\vspace{3cm}

\item Sebuah benda diletakkan di muka cermin cekung yang mempunyai jarak titik api 15 cm. Agar bayangan yang terbentuk 3 kali lebih besar dan nyata, maka benda harus diletakkan di depan cermin sejauh??
\pilgani{
	\item 10 cm
	\item 15 cm
	\item 20 cm
	\item 30 cm
	\item 45 cm
}
\vspace{2cm}

\item Sebuah lensa cembung yang terbuat dari suatu kaca berindeks bias 1,5 memiliki jarak fokus 2,5 cm di udara. jika lensa itu dicelupkan k dalam zat cair yang berindeks bias 1,3. Hitunglah jarak fokus lensa dalam cairan itu
\pilgani{
	\item 2,8 cm
	\item 4,2 cm
	\item 5,6 cm
	\item 8,1 cm
	\item 6,6 cm
}
\vspace{3cm}

\item Bayangan yang terbentuk oleh lensa positif dari sebuah benda yang terletak pada jarak lebih besar dari f tetapi lebih kecil dari 2f dari lensa tersebut (f = jarak fokus lensa) bersifat . . . 
\pilgani{
\item Nyata, tegak, diperbesar
\item Nyata, terbalik, diperbesar
\item Nyata, tegak, diperbesar
\item Nyata, tegak diperkecil
\item Nyata, tegak, sama besar
}
 


\end{enumerate}
\end{multicols*}\end{document}







\documentclass[10pt,a4paper]{extarticle}
\usepackage[latin1]{inputenc}
\usepackage{amsmath}
\usepackage{microtype}
\usepackage[none]{hyphenat}
\usepackage{verbatim}
\usepackage{amsfonts}
\usepackage{amssymb}
\usepackage{enumitem}
\renewcommand{\familydefault}{\sfdefault}
\usepackage{mathpazo}
\renewcommand{\rmdefault}{put}
\usepackage{enumitem}
\usepackage[dvipsnames,svgnames]{xcolor}
\usepackage{tkz-euclide}
\usetkzobj{all}
\usepackage{graphicx}
\usepackage{tikz} 	
\usepackage{adjustbox}
\usepackage{multicol}
\usepackage{lipsum}
\usepackage[left=0.7cm,right=1cm,top=1cm,bottom=1.5cm]{geometry}
\usepackage{cancel} \usepackage{xcolor}
\usepackage{tcolorbox}
\usetikzlibrary{decorations.pathmorphing,patterns}
\usetikzlibrary{decorations.pathreplacing,calc}
 \newcommand\coret[2][red]{\renewcommand\CancelColor{\color{#1}}\cancel{#2}}
\SetLabelAlign{Center}{\hfil\makebox[1.0em]{#1}\hfil}

%%_------= solusi


% Set this =0 to hide, =1 to show

% Set this =0 to hide, =1 to show
\newtcolorbox{mybox}[1][] { colframe = blue!10, colback = blue!3,boxsep=0pt,left=0.2em, coltitle = blue!20!black, title = \textbf{jawab}, #1, } 

%---------- kunci (jika 1 ) muncul
\def\tampilkunci{1}
\newcommand{\hide}[1]{\ifnum\tampilkunci=1
%
\begin{mybox}
 #1
\end{mybox}
%
\vspace{\baselineskip}\fi}



\newcommand*\cicled[1]{\tikz[baseline=(char.base)]{\node[white, shape=circle, fill=red!80,draw,inner sep=0.5pt](char){#1};}}

\newcommand*\kunci[1]{\ifnum\tampilkunci=1
%
\tikz[baseline=(char.base)]{\node[red, shape=circle,draw,inner sep=0.5pt,xshift=2pt](char){#1};}\stepcounter{enumii}
\fi\ifnum\tampilkunci=0
%
\hspace{3pt}#1\stepcounter{enumii}
%
\fi}

\newcommand*\silang[1]{\tikz[baseline=(char.base)]{
\draw[red,thick](-0.2,-0.20)--(0.2,0.2);
\draw[red,thick](-0.2,0.20)--(0.2,-0.2);
\node[black](char){#1};
}}

\newcommand*\centang[1]{\tikz[baseline=(char.base)]{
\draw[red, very thick](-0.2,0.1)--(-0.1,0)--(0.2,0.3);
\node(char){#1};
}}

\newcommand*\merah[1]{
\textcolor{red}{#1}}
\newcommand*\pilgan[1]{
\begin{enumerate}[label=\Alph*., itemsep=0pt,topsep=0pt,leftmargin=*,align=Center] #1 
\end{enumerate}}
\newcommand*\pernyataan[1]{
\begin{enumerate}[label=(\arabic*), itemsep=0pt,topsep=0pt,leftmargin=*] #1 
\end{enumerate}}

\newcommand{\pilgani}[1]{                            \vspace{-0.3cm}\begin{multicols}{2}
 \begin{enumerate}[label=\Alph*., itemsep=0pt,topsep=0pt,leftmargin=*,align=Center]#1                     \end{enumerate}
 \phantom{ini cuma sapi, wedus, dan ayam}
 \end{multicols}}


\begin{document}


 \textbf{Latihan PAT Gelombang Cahaya} \phantom{ini nama siswa yang aaamengerjakan soal kuis ini }  

\begin{multicols*}{2}\raggedcolumns

\begin{enumerate}
\item Dua buah cermin dipasang berdekatan dengan sudut 30$^o$kemudian di depannya diletakkan sebuah benda, maka banyaknya bayangan yang terbentuk adalah . . .
\pilgani{
	\item 3
	\item 6
	\item 11
	\item 20
	\item 30
}
\vspace{2cm}

\item Sebuah benda diletakkan 9 cm di depan cermin cekung yang memiliki jari-jari kelengkungan 20 cm. Sifat-sifat bayangan benda yang dihasilkan cermin adalah . . . 
\pilgan{
	\item Maya, tegak, diperbesar
	\item Maya, tegak, diperkecil
	\item Nyata, tegak, diperbesar
	\item Nyata, terbalik, diperbesar
	\item Nyata, terbalik, diperkecil
}
\vspace{4.2cm}

\item Dua cermin yang masing-masing panjangnya 1,6 m disusun berhadapan seperti pada gambar di bawah ini. Jarak antara cermin 20 cm. Suatu berkas sinar jatuh tepat pada salah satu ujung cermin dengan sudut 30$^o$. Sinar akan keluar dari pasangan cermin setelah mengalami pemantulan sebanyak . . . 
\pilgani{
	\item 16
	\item 13
	\item 15
	\item 9
	\item 4
}
\vspace{4cm}

\item Indeks bias udara besarnya 1, indeks bias air 4/3, dan indeks bias bahan suatu lensa tipis 3/2. Suatu lensa tipis yang kekuatannya di udara 4 dioptri di dalam air akan menjadi . . . 
\pilgani{
	\item 3/5 dioptri
	\item 1 dioptri
	\item 5/2 dioptri
	\item 5/4 dioptri
	\item 5/3 dioptri
}
\vspace{4.2cm}

\item Sebuah benda diletakkan di muka cermin cekung yang mempunyai jarak titik api 15 cm. Agar bayangan yang terbentuk 3 kali lebih besar dan nyata, maka benda harus diletakkan di depan cermin sejauh??
\pilgani{
	\item 10 cm
	\item 15 cm
	\item 20 cm
	\item 30 cm
	\item 45 cm
}
\vspace{2cm}

\item Sebuah lensa cembung yang terbuat dari suatu kaca berindeks bias 1,5 memiliki jarak fokus 2,5 cm di udara. jika lensa itu dicelupkan k dalam zat cair yang berindeks bias 1,3. Hitunglah jarak fokus lensa dalam cairan itu
\pilgani{
	\item 2,8 cm
	\item 4,2 cm
	\item 5,6 cm
	\item 8,1 cm
	\item 6,6 cm
}
\vspace{4.2cm}

\item Bayangan yang terbentuk oleh lensa positif dari sebuah benda yang terletak pada jarak lebih besar dari f tetapi lebih kecil dari 2f dari lensa tersebut (f = jarak fokus lensa) bersifat . . . 
\pilgan{
\item Nyata, tegak, diperbesar
\item Nyata, terbalik, diperbesar
\item Nyata, tegak, diperbesar
\item Nyata, tegak diperkecil
\item Nyata, tegak, sama besar
}
 
 \vspace{4.2cm}

\item Diketahui jarak dua celah ke layar 1,5 m dan panjang gelombang yang digunakan $4\times10^{-7}$ m.
Jarak antara terang pusat dan terang ketiga 0,6 cm. Jarak antara kedua celah adalah . . . .
\pilgani{
\item $3\times 10^{-5}$ m
\item $4,5\times 10^{-5}$ m
\item $1\times 10^{-4}$ m
\item $2\times 10^{-4}$ m
\item $3\times 10^{-4}$ m
}
\vspace{4.2cm}

\item Cahaya monokromatik dengan panjang gelombang 600 nm jatuh pada celah ganda. Jarak layar
terhadap celah sejauh 100 cm. Jika jarak antara terang pusat dengan gelap pertama 2 mm, maka jarak kedua celah adalah . . .
\pilgani{
\item 1,25 mm
\item 0,80 mm
\item 0,60 mm
\item 0,45 mm
\item 0,15 mm }
\vspace{4.2cm}

\item Sebuah celah gandisinari dengan cahaya yang panjang gelombangnya 640 nm. Sebuah layar
diletakkan 1,5 m dari celah. Jika jarak kedua celah 0,24 mm maka jarak dua pita terang yang berdekatan adalah . . .
\pilgani{
\item 4,0 mm
\item 6,0 mm
\item 8,0 mm
\item 9,0 mm
\item 9,6 mm
}
\vspace{4.2cm}

\item Jika panjang gelombang berkas cahaya 600 nm, dan jarak antar
kisi 0,6 mm, jarak celah menuju layar 80 cm, maka jarak antara terang pusat dengan gelap pertama pada layar adalah . . .
\pilgani{
\item 0,2 mm
\item 0,4 mm
\item 0,6 mm
\item 0,9 mm
\item 1,2 mm
}
\vspace{ 3cm}

\item Suatu berkas sinar sejajar yang panjang gelombang 6.000 A mengenai tegak lurus suatu
celah sempit yang lebarnya 0,3 mm. Jarak celah ke layar 1 m. Jarak garis terang pertama ke
pusat pola pada layar adalah . . .
\pilgani{
\item 0,3 mm
\item 0,5 mm
\item 1,0 mm
\item 2,0 mm
item 3,0 mm
} \vspace{4.2cm}


\item  Cahaya monokromatik dari sumber yang jauh datang pada sebuah celah tunggal yang lebarnya
0,80 nm. Pada sebuah layar 3,00 m jauhnya, jarak terang pusat dari pola difraksi ke gelap
pertama sama dengan 1,80 mm. Cahaya tersebut memiliki panjang gelombang . . . 
\pilgani{
\item 320 nm
\item 480 nm
\item 550 nm
\item 600 nm
\item 900 nm
}
\vspace{4.2cm}

\item Sebuah kisi difraksi terdiri atas 5.000 celah per cm. Diketahui spektrum orde terang kedua
membentuk sudut 30$^o$, maka panjang gelombang cahaya yang dijatuhkan pada kisi adalah
\pilgani{
\item 250 nm
\item 300 nm
\item 400 nm
\item 450 nm
\item 500 nm}
\vspace{4.2cm}

\item Seberkas cahaya monokromatik dengan panjang gelombang 500 nm tegak lurus pada kisi
difraksi. Jika kisi memiliki 400 garis tiap cm dan sudut deviasi sinar 30$^o$, maka banyaknya garis terang yang terjadi pada layar 
adalah . . .
\pilgani{
\item 24
\item 25
\item 26
\item 50
\item 51}
\vspace{4.2cm}
\item Cahaya monokromatik dari sumber cahaya datang pada sebuah celah ganda yang lebar antar
celahnya 0,8 mm dan jarak pusat terang ke terang kedua adalah 1,80 mm dan panjang gelombang adalah
4800A maka jarak celah ke layar adalah . . 
\pilgani{
\item 2 m
\item 1,5 m
\item 1 m
\item 0,5 m
\item 0,02 m
}
\vspace{4.2cm}
\item Suatu cahaya tak terppolarisasi mengenai polaroid pertama dengan intensitas $I_o$.
Maka intensitas cahaya dari polaroid kedua, jika sudut antara kedua sumbu transmisi adalah 30$^o$
adalah . . .
\pilgani{
\item $ \frac{3}{8} I_o$
\item $ \frac{1}{8} I_o$
\item $ \frac{1}{4} I_o$
\item $ \frac{1}{6} I_o$
\item $ \frac{1}{3} I_o$
}
\vspace{4.2cm}
\item Dua keping polarisator disusun sejajar
dengan sumbu transmisi yang sejajar pula.
Cahaya alami (tak terpolarisasi) yang masuk
ke susunan polarisator itu akan mengalami
penurunan intensitas sebanyak 75% jika
polarisator yang kedua diputar . . . . . derajat
\pilgani{
\item 30
\item 37
\item 45
\item 53
\item 60
}\vspace{4.2cm}

\item Suatu gelombang datang dari medium yang berindeks bias $\frac{3}{2}$ menuju medium yang berindeks bias $\frac{3}{4}\sqrt{6}$. Jika besar sudut datang adalah 60$^o$ tentukan besar sudut bias yang terjadi!
\pilgani{
\item 30$^o$
\item 45$^o$
\item 60$^o$
\item 53$^o$
\item 90$^o$
}
\vspace{4.2cm}

\item Kedalaman suatu kolam adalah 12 m. Jika indeks bias air adalah 4/3 maka kedalam semu kolam tersebut
\pilgani{
\item 24 m
\item 16 m
\item 9 m
\item 6 m
\item 3 m
}
\vspace{4.2cm}

\end{enumerate}
\end{multicols*}\end{document}






